\documentclass[../Main.tex]{subfiles}

\begin{document}

\chapter{Metric Spaces}

\section{Fundamental Definitions n' Stuff}
\defn{Metric Space}{A set $X$ along with a function $d:X \times X \to \RR^{+} \cup \{0\}$ called distance, is said to be a metric space if: \begin{enumerate}
    \item $d(x,y)=0 \iff x=y$ (Positivity)
    \item $d(x,y)=d(y,x), \forall x,y \in X$ (Symmetric)
    \item $\forall x,y,z \in X$ we have, $d(x,y) \leq d(x,z)+d(y,z)$ (Triangle Inequality)
\end{enumerate}}
\exm{$\RR^n$ as a metric space}{Note that $\RR^n$, the set of all n-tuples of $\RR$, is a metric space with $$d(\vec x,\vec y):=|\vec x-\vec y|= \sqrt{(x_1-y_1)^2+(x_2-y_2)^+\cdots+(x_n-y_n)^2} $$}
\defn{Open and Closed Balls around $x$ in $X$}{Open ball is defined as: $$B_{r}(x):=\{y \in X:d(y,x)<r \}$$
\\\\ Closed ball is defined as: $$ B_{[r]}(x):=\{y\in X:d(y,x) \leq r\}$$}
\defn{Convexity}{A set $S$ in $\RR^n$ is said to be convex if $\forall x,y \in S, t \in [0,1]$, $x+t(y-x) \in S$}
\exm{Open and closed balls in $\RR^n$ are convex}{Consider $B_r(x):=\{z \in \RR^n: |z-x|<r \}$. Consider arbitrary $p$ and $q$ in $B_r(x)$. We have that $d(p,x)<r$ and $d(q,x)<r$. Consider $p+t(q-p)$ and consider $d(p+t(q-p),x)=|p+t(q-p)-x|=|tq+(1-t)p-x+tx-tx|=|tq-tx+(1-t)p-(1-t)x|\leq t|q-x|+(1-t)|p-x|=td(q,x)+(1-t)d(p,x)<r$. Replacing $<$ with $\leq$ in the above proves the result for closed balls.

}
\defn{Sequences in Metric Spaces}{A sequence $\{x_n: x_n \in X\}$ is a mapping from the naturals to $X$, where order is implicit. We say a sequence in a metric space $X$ is convergent to $x \in X$ if: $$(\forall \varepsilon>0)(\exists n_0 \in \NN)(\forall n \in \NN, n \geq n_0)(d(x_n,x)<\varepsilon)$$}
\defn{Cauchy Sequence in $X$}{A sequence $\{x_n\}$ in $X$ is said to be cauchy if $$(\forall \varepsilon>0)(\exists n_0 \in \NN)(\forall n,m \in \NN)(n,m \geq n_0 \implies d(x_n,x_m)<\varepsilon) $$}
\thmp{}{Convergence $\implies$ cauchy}{Say $x_n$ converges to $x$ in the metric space. $$(\forall \varepsilon>0)(\exists n_0)(\forall n,m \geq n_0)(d(x_n,x)<\varepsilon, d(x_m,x)<\varepsilon)$$ which implies (from triangle inequality)
$$(\forall \varepsilon>0)(\exists n_0)(\forall n,m \geq n_0)(d(x_n,x)<\varepsilon, d(x_m,x_n)\leq d(x_m,x)<\varepsilon)$$
Which is the cauchy condition.}
\newpage
\subsection{Inner products, Normz and some common metrics}
\defn{Inner Product}{Let $\FF$ be either $\CC$ or $\RR$, and $V$ a vector space over $\FF$. An \textbf{inner product} on $V$ is a function that assigns to each ordered pair of vectors $v,u \in V$ a scalar $\langle v|u\rangle \in \FF$ so that 
the following holds:

\begin{enumerate}
    \item $\langle v+cw|u \rangle=\langle v|u \rangle+c\langle w|u\rangle$
    \item $\langle u|v \rangle=\overline{\langle v|u \rangle}$ where $\overline{\cdot}$ is just complex conjugation
    \item $\langle u|u \rangle>0$ if $u \neq 0$
\end{enumerate}
A space $V(\FF), \langle \cdot \rangle $ is called an \emph{inner product space}. 
}
\defn{Norm (From the inner product)}{Given an inner product space $V(\FF), \langle \cdot \rangle$, we define the \emph{norm} or \emph{length} of a vector $x$, given by $|| x||$ as $\sqrt{\langle x|x \rangle}$ }
\exm{$p$-norm is a metric}{The $p$-norm is defined on $\RR^n$ as: $$||\vec x ||_p:=(|x_1|^p+|x_2|^p \cdots |x_n|^p)^{1/p}=(\sum_{i=1}^n |x_i|^p)^{1/p} $$
We say $d_p(\vec x,\vec y)$, the distance between $\vec x$ and $\vec y$ to be $||x-y||_p=(\sum_{i=1}^n |x_i-y_i|^p)^{1/p}$. From this definition, unique $0$ distance property and reflexivity property are obvious. Consider $d_p(x,z)$ and $d_p(z,y)$. $d_p(x,y)=(\sum_{i=1}^k|(x_i-z_i)+(z_i-y_i)|^p)^{1/p}$ whence from Minkowski we see that $$(\sum_{i=1}^k|(x_i-z_i)+(z_i-y_i)|^p)^{1/p}
\leq (\sum_{i=1}^n |x_i-z_i|^p)^{1/p}+(\sum_{i=1}^n |z_i-y_i|^p)^{1/p}=d_p(x,z)+d_p(y,z)$$
Therefore, $p$-norm is a metric 
}

\exm{The Max Norm}{Suppose $x \in \RR^n$, the max norm $||x||_{\infty}$ is defined as: $$||\vec x||_{\infty}:= \max_{1 \leq i \leq n}(|x_i|)$$
Clearly, $d_{\infty}(x,y)=0$ if and only if $x=y$. Moreover, from the properties of $|\cdot|$, it is reflexive. Consider $d_{\infty}(x,y)=\max\{|x_1-y_1|,|x_2-y_2|\cdots,|x_n-y_n| \}, d_{\infty}(x,z)=\max\{|x_1-z_1|,|x_2-z_2|\cdots,|x_n-z_n| \}$ and $d_{\infty}(y,z)=\max\{|y_1-z_1|,|y_2-z_2|,\cdots |y_n-z_n| \}$. 
$$d_{\infty}(x,y)=\max_{1\leq 1\leq n}\{|(x_i-z_i)+(z_i-y_i|)\} \leq \max_{1\leq 1\leq n}\{|(x_i-z_i)\}+\max_{1\leq 1\leq n}\{|(y_i-z_i)\}=d_{\infty}(x,z)+d_{\infty}(z,y)$$
Hence the max norm is a metric.
}

\exm{Norm on function spaces}{Let $X$ be a non empty set. Define $\mathfrak{B}(X)$ as the space of all bounded, real functions. Then $||f||_{\infty}:=\sup_{x\in X}|f(x)|$ defines a norm on $\mathfrak{B}(X)$ since $|f(x)-g(x)| \leq |f(x)|+|g(x)|\leq \sup_{x \in X}|f(x)|+\sup_{x \in X}|g(x)|=||f||_{\infty}+||g||_{\infty}$ which gives us $||f+g||_{\infty}\leq ||f||_{\infty}+||g||_{\infty}$}

\exm{the $\frac{d}{1+d}$ metric}{Say $X,d$ is a metric space, then $g(x,y)=\frac{d(x,y)}{1+d(x,y)}$ is a metric.
\\\\ \pf{exercise to the reader}
}
\exm{Constructing a metric from an injective function}{Let $Z$ be a set and $(X,d)$ a metric space. Let $f:Z \to X$ be an injective function. Define $\delta(x,y)=d(f(x),f(y))$. This defines a metric on $Z$ and makes $(Z,\delta)$ a metric space since $\delta(x,x)=d(f(x),f(x))=0$, $\delta(x,y)=0\implies d(f(x),f(y))=0$ which gives $x=y$, and $\delta(x,y)=d(f(x),f(y))\leq d(f(x),f(z))+d(f(z),f(y))=\delta(x,z)+\delta(y,z)$ which gives triangle. } 
\exm{The discrete metric}{Let $X$ be any set whatsoever. Define the metric $$d(x,y):=\begin{cases} 1 \ \text{if } x\neq y \\ 0 \text{ if } x=y \end{cases}$$ This is a metric due to positive definiteness and symmetry. The triangle inequality is, also, not hard to see.}
\exm{Space of all bounded sequences }{Let $X$ be the set of all bounded sequences of reals, i.e, the set of all $x$ such that $x=(x_1,x_2,\cdots)=\{x_n\}_{n=1}^{\infty}$ with $\sup_{i=1}^{n}(\{x_i\})<\infty$. If $\{x_i\}$ and $\{y_i\}$ are sequences (i,e, elements in $X$) let the distance between two sequences be defined by $$d(x,y):=\sup_{i=1}^{\infty}|x_i-y_i|$$
Clearly, it is positive definite and symmetric. Consider three sequences $\{x_n\}$, $\{y_n \}$ and $\{z_n\}$ in $X$. $d(x,y)=\sup\{|x_i-y_i|:i \in \NN\}$, $d(x,z)=\sup\{|x_i-z_i|:i \in \NN\}$ and $d(y,z)=\sup\{|y_i-z_i|:i \in \NN\}$. Cosnider $d(x,z)=\sup\{|x_i-z_i|=|x_i-y_i+y_i-z_i|:i \in \NN\} \leq \sup\{|x_i-y_i|+|y_i-z_i|:i \in \NN\}$ $\implies \leq \sup\{|x_i-y_i|:i \in \NN\}+\sup\{|y_i-z_i|:i \in \NN\}$ which proves the triangle. }


\subsubsection{Basics on Continuity (Read definition of sequence in metric space, and limit points)}
\defn{Continuous functions}{A function $f:X \to Y$ that maps a metric space $X,d_x$ to a metric space $Y,d_y$ is said to be continuous at $a \in X$ (which is a limit point of $X$) if $$(\forall \varepsilon>0)(\exists \delta>0)(\forall p \in X)(d_x(p,a)<\delta \implies d_y(f(p),f(a))<\varepsilon)$$}
\thmp{Sequential Criteria for continuity}{Let $X,d_x$ and $Y,d_y$ be metric spaces and $f:X \to Y$. Let $a$ be a limit point of $X$. Then $f$ is continuous at $a$ if and only if for every sequence $(q_n) \in X$, $q_n \neq a$ with $\lim(q_n)=a$, we have $lim(f(q_n))=f(a)$}{$\implies$) We have that for every $\varepsilon>0$, there exists a $\delta>0$ so that for all $z \in X$ such that $z \in B_{\delta}(a)$, we have $f(z) \in B_{\varepsilon}(f(a))$. Consider an arbitrary sequence in $X$ that is so that $(q_n) \in X$, $q_n \neq a$ and $\lim(q_n)=a$. Which reads $\forall \delta>0, \exists n_0 \in \NN$ so that $\forall n \geq n_0$, we have $q_n \in B_{\delta}(a)$. Let $\varepsilon>0$ be arbitrary. From the definition of continuity, there exists a corresponding $\delta>0$ so that $\forall z \in X$, $z \in B_{\delta}(a)$ would imply $f(z) \in B_{\varepsilon}(f(a))$. For this $\delta_{\varepsilon}$, there exists an $n_0(\delta)(\varepsilon) \in \NN$ so that $\forall n \geq n_0$, $q_n \in B_{\delta}(a)$ which would imply $f(q_n) \in B_{\varepsilon}(f(a))$. This means that $\forall \varepsilon>0$, there exists $n_0 \in \NN$ so that $\forall n \geq n_0$, we have $f(q_n) \in B_{\varepsilon}(f(a))$ which means that $\lim(f(q_n))=f(a)$.
\\\\ $\impliedby$) Suppose that every sequence $q_n \in X$ so that $q_n \neq a$ with $\lim(q_n)=a$, we have that $\lim(f(q_n))=f(a)$ but that the function $f$ is \emph{not} continuous. i.e, $\exists \varepsilon>0$ so that $\forall \delta>0$, $\exists q_{\delta} \in X$ so that $d_x(q_{\delta},a)<\delta$ but $d_y(f(q_{\delta}),a)\geq \varepsilon$. Choose $\delta=1$ and get the corresponding $x_1 \in B_{1}(a)$ for which we have $d_y(f(x_1),f(a))\geq \varepsilon$. Choose $\delta=1/2$ and get $x_2 \in X$ so that $d_x(x_2,a)<1/2$ but $d_y(f(x_2),f(a))\geq \varepsilon$. As such, keep going with $\delta=1/n$ to generate a sequence $x_n \in X$ so that $d(x_n,a)<1/n$ but $d(f(x_n),f(a))\geq \varepsilon$. Note that $x_n$ converges by definition, to $a$, but there exists an $\varepsilon>0$ so that no matter what $n_0$ we take, there exists some $n \geq n_0$ so that $d_y(f(x_n),f(a))\geq \varepsilon$ which means that the sequence $f(x_n)$ does not converge to $f(a)$, which contradicts hypothesis. }
\corp{Composition of continuous functions (at a point $a$) is finally a continuous function}{Suppose $X,d_x$,$Y,d_y$ and $Z,d_z$ are metric spaces with $f:X \to Y$ and $g:Y \to Z$ two continuous functions. $g \circ f: X \to Z$. Consider a sequence $(x_n) \in X$ that converges to $a$. From sequential criteria, $f(x_n) \to f(a)$. We have $f(x_n)$, a sequence in $Y$, that converges to $f(a)$. From the continuity of $g$, we note that $g \circ (f(x_n))$ converges to $g \circ(f(a))$, which ultimately tells that for every sequence $\{x_n\}$ in $X$ that converges to $a$, $g \circ f(x_n)$ is a sequence in $Z$ that converges to $g \circ f (a)$ whence we are done}{}

\exm{Dirichlet Function}{Let $f:\RR \to \RR$ be defined by $$f(x)=\begin{cases}
1 \ \text{ if } x \text{ is rational} \\
0 \ \text{ if } x \text{ is irrational}
\end{cases}$$ This function is discontinuous at every point of $\RR$. Bear in mind that to show that a function is discontinuous at a point $z$, we need only show that \textit{one} sequence in $A$ which converges to $z$ is such that $\lim(f(x_n))\neq f(\lim(x_n))$. Consider $a \in \RR$ an irrational point. Consider the rational sequence $x_n$ that converges to $a$ via density theorem. $f(x_n)=1$ for every $n \in \NN$ which means $\lim f(x_n)=1$. But $f(\lim(x_n))=0$. Hence, discontinuous at $a \in \RR$, irrational. In fact, the same argument can be re-used to show that $f$ is irrational at every point.}

\exm{Thomae Function}{We define the Thomae function as follows: Let $f:\RR^+ \to \RR$, 
$$f(x):=\begin{cases}
    0 \ \text{ if } x \text{ is irrational} \\
    {1/n} \text{ if } x \text{ is rational, with }x=m/n, gcd(m,n)=1
\end{cases}$$
This function is discontinuous at rational points quite obviously since, consider a rational number $m/n$ where $f(m/n)=1/n>0$. Consider an irrational sequence in $\RR^+$ converging to $m/n$. Clearly, $f(x_n)=0$ for every $n \in \NN$ whence, it is clear.
\\\\ Consider $z$ an irrational point where $f(z)=0$. Consider a $\varepsilon>0$ and an $n_0$ so that $1/n_0<\varepsilon$. Consider the subinterval $(z-\gamma,z+\gamma)$ for some $\gamma$. We note that in this subinterval, only finite rational points have their denominator smaller than $n_0$ (because $(z-\gamma)n<m<(z+\gamma)n$ and this interval has size $2n\gamma$. For a fixed $n$ and $\gamma$, $m$ can pick values only from this interval. If $n<n_0$, then the interval size is smaller than $2n_0(\gamma)$. If we make $\gamma$ small enough, $m$ would be restricted to just one value. say, $m_0$. With this $\gamma$ (and $m_0$ fixed) the only values $n$ can take that are less than $n_0$ are certainly finite now since there are only finite $n$s smaller than $n_0$. Summarising, for a given $\varepsilon$, we choose an $n_0$ so that $1/n_0<\varepsilon$, where we concluded that for small enough $\gamma$, only finite rational points in the interval $(z-\gamma,z+\gamma)$ have their denominator smaller than $n_0$). Now, we can eliminate all these rational points by picking an appropriate $\delta$ so that no rational points with denominator smaller than $n_0$ exist in $(z-\delta,z+\delta)$. Therefore, for a given $\varepsilon$ ($\exists n_0: 1/n_0<\varepsilon$) there exists $\delta$ so that either a point $q$ in $(z-\delta,z+\delta)$ is irrational, whence $f(q)=0 <\varepsilon$, or it has its denominator larger than $n_0$, which means $f(q)=1/n<1/n_0<\varepsilon$ which leads to $|f(q)-f(z)|=|f(q)|=f(q)<\varepsilon$ which is the "$\varepsilon-\delta$" criterion for Continuity. Hence, at all irrational points $z$, the Thomae function is continuous.
}
Now we deal with, primarily, continuous functions on closed, bounded intervals of the kind $\II=[a,b] \subset \RR$.
\thmp{Boundedness Theorem}{Continuous functions on closed bounded intervals are bounded}{Suppose $f:\II \to \RR$ is actually unbounded, i.e, for any $M \in \RR$ we take, there exists an $x_M \in \II$ so that $f(x_M)>M$. Let $M_1=1$ and get the corresponding $x_1$ so that $f(x_1)>1$. Do the same for $M_n=n$ to get $x_n$ so that $f(x_n)>n$. This sequence $f(x_n)$ is divergent. But the sequence $\{x_n\}$ is bounded, hence from Bolzano, there is a subsequence $x_{n_k}$ that is convergent, say to $p$. But $f(x_{n_k})$ is a subsequence of $f(x_n)$, which means it diverges. But according to continuity, $f(x_{n_k})$ converges to $f(p)$, which is a contradiction. Hence, $f$ is bounded. }
\thmp{Maxima-Minima Theorem}{Let $f:\II \to \RR$ be a continuous function on closed bounded interval $\II$. Then $f$ attains its maxima and minima.}{We know that $f(\II)$ is actually bounded, which means it has supremum and an infimum $U$ and $L$. Suppose it \emph{does not} attain bounds, i.e, (talking about upper bound) $\forall x \in \II$, $f(x)<U$. Consider $U-1$. There exists $f(x_1)$ so that $U-1<f(x_1)<U$. Choose $x_2$ likewise so that $U-1/2<f(x_2)<U$. Keep going as such, to find $x_n$ so that $U-1/n<f(x_n)<U$. This means from squeeze theorem that $f(x_n)$ converges to $U$. Note that $x_n$ that we have collected is a sequence that is in $\II$, a bounded, closed interval. Hence, it has a convergent subsequence $x_{n_k} \to x$. Also, $f(x_{n_k})$ is a subsequence of $f(x_n)$ which means $f(x_{n_k}) \to U$. But from continuity, $f(x_{n_k})\to f(x)$, which implies $f(x)=U$, where $x \in \II$. This means that the function actually \emph{does} attain upper bound (similar argument for the lower bound can be performed).}
\thmp{Location of roots theorem}{Let $f:\II \to \RR$ be a continuous function on a closed bonunded interval $\II$. Let $a<b$ with $f(a)<0$ and $f(b)>0$ (or the other way around). Then there exists $c \in (a,b)$ so that $f(c)=0$}{Let $a_0=a$ and $b_0=b$ with $a<b$. $a_0,b_0$ forms an interval of size $\xi=b_0-a_0$. Look at $f(a_0+b_0/2)$. Is it greater than $0$? If so, make $a_1=a_0$, and $b_1=a+b/2$, a new interval $a_1,b_1$ of size $\xi/2$. This new interval obeys the property that $f(a_1)<0$ and $f(b_1)>0$. Is $f(a_0+b_0/2)<0$? If so, make $b_1=b_0$ and $a_1=a_0+b_0/2$ to make $a_1,b_1$ a new interval of size $\xi/2$. Again, $f(a_1)<0$ and $f(b_1)>0$ Suppose you have made the $n-th$ interval $a_n,b_n$ so that $f(a_n)<0$ and $f(b_n)>0$ of size $\xi/(2^n)$. Make $a_{n+1},b_{n+1}$ by asking similar questions as above. We thus have a sequence of nested intervals $[a_n,b_n]$, for which $f(a_n)<0$ and $f(b_n)>0$. From Nested interval theorem, we have $\gamma \in \cap_n [a_n,b_n]$, moreover, since the size of these intervals are converging to $0$, this is a unique point in the intersection. $a_n \to \gamma$ and $b_n \to \gamma$. From continuity, then, we have $f(a_n) \to f(\gamma) \leq 0$ and $f(b_n)\to f(\gamma) \geq 0$ which means that $f(\gamma)=0$. }
\thmp{Bolzano's IVT}{Let $\II$ be a closed, bounded interval and let $f:\II \to \RR$ be a continuous function. If $a,b \in \II$ with $k \in \RR$ satisfying $f(a)<k<f(b)$, then there is a point $c $ in $\II$ so that $f(c)=k$}{Let $a<b$ and define $g(x)=f(x)-k$. This is a continuous mapping on $\II$. We then have $g(a)<0<g(b)$ whence from location or roots theorem, we find $c \in \II$ so that $g(c)=0$ or $f(c)=k$. }
\thmp{Preservation of intervals Theorem}{Let $\II$ be a closed, bounded interval. Let $f:\II \to \RR$ be a continuous map. Then, $f(I)$ is also a closed bounded interval.}{Consider $f(\II):=\{ z \in\RR: \exists x \in \II \text{ such that }f(x)=z\}$. We know that continuous functions attain bounds, i.e, $f(\II)$ has a maxima and a minima that will (eventually) be the end points of our closed bounded interval. If the function is constant, we are done. Consider a point $\min(f(\II))<z<\max(f(\II))$. By Bolzano's IVT, we know that a pre-image exists for $z$. This means that $z$ is in the image. We are, therefore, done. The requisite interval is $[\min(f),\max(f)]$.}
\defn{Uniform Continuity}{A function $f:A \to \RR$ is said to be uniformly continuous on $A$ if for every $\varepsilon>0$, $\exists \delta>0$ so that for all $x,u \in A$ such that $0<|x-u|<\delta$, we have $|f(x)-f(u)|<\varepsilon$.
\\\\ We see that this definition is similar to the definition of continuity on $A$ except for the fact that $\delta$ is independent of the $x$, the point at which we speak of continuity.}
\thmp{}{Let $f$ be a function defined on $A \subseteq \RR$. The following are equivalent:
\begin{enumerate}
    \item $f$ is not uniformly continuous on $A$
    \item $\exists \varepsilon>0$ such that $\forall \delta>0$, $\exists x_{\delta},u_{\delta} \in A$ so that $0<|x_{\delta}-u_{\delta}|<\delta$ but $|f(x_{\delta})-f(u_{\delta})|\geq \varepsilon$
    \item $\exists \varepsilon$ and two sequences $x_n$ and $y_n$ so that $lim(x_n-u_n)=0$ but $|f(x_n)-f(u_n)|\geq \varepsilon$ for every $n \in \NN$
\end{enumerate}
}{$(1)\implies(2)$) Obvious negation of the definition
\\\\ $(2)\implies(3))$ Choose $\delta=1$, and get corresponding $x_1, u_1$ so that $0<|x_1-u_1|<1$ and $|f(x_1)-f(u_1)|\geq \varepsilon$. Choose $\delta=1/2$ and get the corresponding $x_2$ and $u_2$ so that $0<|x_2-u_2|<1/2$ and $|f(x_2)-f(u_2)|\geq \varepsilon$. Keep going as such to get $x_n,u_n$ so that $0<|x_n-u_n|<1/n$ $\forall n \in \NN$ and $|f(x_n)-f(u_n)|\geq \varepsilon$ for every $n \in \NN$. This means that $\lim(x_n-u_n)=0$ but $|f(x_n)-f(u_n)|\geq \varepsilon$ for every $n\in \NN$.
\\\\ $(3) \implies (2))$ $\forall \delta>0$, $\exists n_0(\delta)$ so that $\forall n \geq n_0$, we have $0<|x_n-u_n|<\delta$ but $|f(x_n)-f(u_n)|\geq \varepsilon$ , choose one of these $x_n-$s for $n$ greater than $n_0(\delta)$ as our $x_{\delta}$(and likewise for $u_{\delta}$). We are then done. }
\thmp{Uniform Continuity Theorem}{Let $\II$ be a closed bounded interval and let $f:\II \to \RR$ be continuous on $\II$. Then $f$ is uniformly continuous on $I$}{Suppose $f$ is not uniformly continuous on $\II$. From the previous \emph{non uniform} criteria, we have that $\exists \varepsilon$ and two sequences in $\II$, $x_n, u_n$ so that $\lim(x_n-u_n)=0$ (or $\lim(x_n)=\lim(u_n)$ ) but $|f(x_n)-f(u_n)|\geq \varepsilon$ for every $n \in \NN$. Since $x_n$ and $u_n$ are sequences in $\II$ which is closed and bounded, $x_n$ has a subsequence $x_{n_k}$ converging to $x$. Since $\lim(x_n-u_n)=0$, $\lim(x_{n_k}-u_{n_k})=0$ Which means that $\lim(u_{n_k})=x$ as well. We have $x_{n_k} \to x$ and $u_{n_k}\to x$, which means from continuity that $f(x_{n_k})\to f(x)$ and $f(u_{n_k})\to f(x)$, whence we find that $\lim(f(x_{n_k})-f(u_{n_k}))=0$ necessarily. But this contradicts the assumption that $f$ is not uniform continuous. Hence, we are done.}
\subsubsection{Sequences of Functions}
Given a set $A \subset \RR$, we primarily work with a (countably) infinite collection of functions $\{f_n\}$ with $f_n:A \to \RR$. We generate a sequence of numbers by evaluating $f_n$ at a fixed point $x \in A$. It could be, then, that $\{f_n(x)\}$, treated as a sequence of numbers in $\RR$, either converges or does not. For a subset $A_0 \subseteq A$ (which can possibly be empty), $f_n(x)$ converges for every $x \in A_0$. If this happens, there is a uniquely determined value that we would like to call "$f(x)$". Thus, there arises naturally a function $f:A_0 \to \RR$ that we call the "limit" of $\{f_n\}$.
\defn{Convergence of sequence of functions}{Let $\{f_n\}$ be a sequence of functions defined on $A \subseteq \RR$. Let $A_0 \subset A$, and let $f:A_0 \to \RR$. We say the sequence $(f_n)$ converges to $f$ \textbf{pointwise} if for each $x \in A_0$, The sequence (of real numbers) $f_n(x)$ converges to $f(x)$.
\\\\ In other words, $\{f_n:A \to \RR\}$ converges to $f$ pointwise if for every $x \in A_0$, $\forall \varepsilon>0$, $\exists n_0(x,\varepsilon) \in \NN$ such that $\forall n \geq n_0(x,\varepsilon)$, we have $|f_n(x)-f(x)|<\varepsilon$}
If we remove the dependence of $n_0$ above on the $x$, and leave it just to depend on $\varepsilon$, we arrive at the definition of \emph{Uniform Convergence}.
\defn{Uniform Convergence of a sequence of functions}{Let $\{f_n\}$ be a sequence of functions defined on $A \subset \RR$. We say $\{ f_n\}$ converges \textbf{uniformly} to $f:A_0 \to \RR$ if $\forall x \in A_0$, $\forall \varepsilon>0$, $\exists n_0(\varepsilon) \in \NN$ so that $\forall n \geq n_0(\varepsilon)$, $|f_n(x)-f(x)|<\varepsilon$.
\\\\ Since the dependence of $n$ on $x$ is non-existent, we can rewrite the above definition to read:
$\{ f_n\}$ converges \textbf{uniformly} to $f:A_0 \to \RR$ if $\forall \varepsilon>0$, $\exists n_0(\varepsilon) \in \NN$ so that $\forall n \geq n_0(\varepsilon)$, $\forall x \in A_0$, $|f_n(x)-f(x)|<\varepsilon$.
}

\thmp{Non uniform convergence criteria}{A sequence of functions $\{f_n\}$ defined on $A$ \textbf{does not} converge \textbf{uniformly} to $f:A_0 \to \RR$ if and only if $\exists \varepsilon>0$ such that $\forall k \in \NN$, $\exists n_k \geq k$ and $x_k \in A$ so that $|f_{n_k}(x_k)-f(x_k)|\geq \varepsilon$, which means for some $\varepsilon>0$, there exists a subsequence $f_{n_k}$ of $f_n$ and a sequence in $x_k$ in $A_0$ so that $$|f_{n_k}(x_k)-f(x_k)|\geq \varepsilon\ \forall k \in \NN$$}{Obvious}

\defn{The Uniform Norm}{Say $A \subseteq \RR$ and $\phi:A \to \RR$. We say $\phi$ is bounded on $A$ if $\phi(A)$, the set, is bounded in $\RR$. If it is bounded, we can define what is \textbf{the uniform norm of $\mathbf{\phi}$ on $A$} by: 
$$||\phi||_A:=\sup\{|\phi(x)|:x \in A\} $$
\\\\ It follows that $||\phi||_A \leq \varepsilon \iff |\phi(x)|\leq \varepsilon, \forall x \in A$
}

\lemp{}{A sequence $f_n$ of bounded functions on $A$ converges uniformly to $f:A \to \RR$ if and only if $||f_n-f||_A \to 0$ }{$\implies$) Say $f_n \overline{\to} f$, this means that $\forall \varepsilon>0, \exists n_0 \in \NN$ so that $\forall n\geq n_0$, $\forall x \in A$, we have $|(f_n-f)(x)|<\varepsilon$, which means that for every $n \geq n_0$, we have $|| (f_n-f)||_A \leq \varepsilon$ which concludes the forward direction.
\\\\ $\impliedby$) Say $||f_n-f||_A \to 0$, which means that $\forall \varepsilon>0, \exists n_0 \in \NN$ so that $\forall n \geq n_0,$ we have $||f_n-f ||_A <\varepsilon$ which means $\sup\{ |(f_n-f)(x)|: x \in A\}<\varepsilon$ which means $\forall x \in A$, $|f_n(x)-f(x)|<\varepsilon$ whence the back implication is also done. 

}

\thmp{Cauchy criteria for uniform convergence}{A sequence $f_n$ of bounded functions on $A$ is uniformly convergent to $f:A \to \RR$ (a bounded function) if and only if for every $\varepsilon>0$, there is an $n_0 \in \NN$ so that $\forall m,n \geq n_0$, $||f_m-f_n||_A \leq \varepsilon$}{$\implies$) Say $f_n \overline{\to} f$. This means that $\forall \varepsilon>0$, $\exists n_0$ so that $\forall n, m \geq n_0$, we have, for every $x \in A$,  $|f_n(x)-f(x)|<\varepsilon/2$ and $|f(x)-f_m(x)|<\varepsilon/2$. Simply add them. We then have $|(f_m-f_n)(x)|<\varepsilon$ for every $x \in A$. In terms of the uniform norm. we see that $\forall \varepsilon>0$ ,$\exists n_0 \in \NN$ so that for all $m,n \geq n_0$, we have $||f_m-f_n||_A \leq \varepsilon$.
\\\\ Suppose that $\forall \varepsilon>0$, $\exists n_0 \in \NN$ so that $\forall m,n \geq n_0$, $||f_m-f_n||_A \leq \varepsilon$ which means that for every $x \in A$, we have $|f_m(x)-f_n(x)|<\varepsilon$. This means that for every $x$, $f_n(x)$ is a cauchy sequence, hence convergent to some $f(x)$. More can be said for the sequence at each $x$. Fix some $x \in A$. We have that $\forall \varepsilon>0$, $\exists n_0 (\varepsilon) \in \NN$ so that $\forall n,m \geq n_0$, we have $|f_n(x)-f_m(x)|<\varepsilon$ which means that $\forall n \geq n_0$, $|f_n(x)-f(x)|<\varepsilon$ which means that, all in all, $\forall \varepsilon>0$, $\exists n_0(\varepsilon) \in \NN$ so that $\forall n \geq n_0$, $\forall x \in A$, we have $|f_n(x)-f(x)|<\varepsilon$ which is the uniform convergence criteria. }

\newpage

\subsubsection{$l_p$ Space}

\defn{$l_p$-Space}{Define $l_p$ as the space of all sequences $(a_n)_{n=1}^{\infty}$ such that $\sum_{i=1}^{\infty} |a_i|^p$ exists. \textbf{The set of all sequences in Real line that are \emph{p summable}}}

It is easily seen thet $l_p$ is a vector space over $\RR$. We defne the metric on $l_p$ as $d_p(\{x\},\{y\})=(\sum_{i=1}^{\infty}|x_i-y_i|^p)^{1/p}$. Positive definiteness is clear since the sum would not be $0$ if even one index is different. Symmetry is obvious, and triangle: $$ d(x,z)=(\sum_{i=1}^{\infty} |x_i-z_i|^p)^{1/p} \leq (\sum_{i=1}^{\infty} |x_i-y_i+y_i-z_i|^p)^{1/p} $$ which gives $$\leq (\sum_{i=1}^{\infty} |x_i-y_i|^p)^{1/p}+(\sum_{i=1}^{\infty} |y_i-z_i|^p)^{1/p} =d(x,y)+d(y,z)$$
(This must be obvious from the infinite case of minkowski inequality).

\exm{Continuous functions on a closed bounded interval}{$C[a,b]$ is a metric space under the metric $d(f,g):=\sup\{|f(x)-g(x)|:x \in [a,b] \}$. Positive definiteness comes from $|\cdot |$. Symmetry is aswell obvious. Look at $d(f,g)=\sup\{|f(x)-g(x)|:x\in [a,b] \}=\sup\{|f(x)-h(x)+h(x)-g(x)|x \in [a,b]\}$ which gives $\leq \sup\{|f(x)-h(x)|+|h(x)-g(x)|: x \in[a,b] \} \leq \sup\{|f(x)-h(x)|:x \in [a,b]\}+\sup\{|h(x)-g(x)|: x \in[a,b]\}$ which gives us our desired $d(f,g)\leq d(f,h)+d(h,g)$}
\exm{Space of all sequences in $\RR$}{We consider the space of all sequences in $\RR$ with the metric $d(\{x\},\{y\}):=\sum_{i=1}^{\infty}\frac{|x_i-y_i|}{(1+|x_i-y_i|)2^i}$. This is quite well defined since the series is bounded by $\sum 1/2^i$. Positive definiteness is clear since the sum is of positive numbers. Symmetry is aswell clear since $d(x,y)/1+d(x,y)$ is a well defined metric as seen before. Consider $d(x,y):=\sum_{i=1}^{\infty} \frac{1}{2^i}(\frac{|x_i-y_i|}{1+|x_i-y_i|})\leq \sum_{i=1}^{\infty} \frac{1}{2^i}(\frac{|x_i-z_i|}{1+|x_i-z_i|})+\sum_{i=1}^{\infty} \frac{1}{2^i}(\frac{|z_i-y_i|}{1+|z_i-y_i|})=d(x,z)+d(z,y)$ (the inequality comes from the metric structure of $d/1+d$ metric). }
\defn{Norm}{Given a Linear space $V(\RR)$, a norm, $||.||$, is a function mapping every vector to a real number satisfying the following for every $x,y \in V$ and every $\alpha \in \RR$:
\begin{enumerate}
    \item $||x|| \geq 0$ and equality of and only if $x=0$.
    \item $||\alpha x||=|\alpha|||x||$
    \item $||x+y|| \leq ||x||+||y||$ (Triangle)
\end{enumerate}
}
\defn{Psuedo Metrics}{Let $X$ be a non-empty set. A psuedometric is a function $d:X \times X \to \RR$ that follows:
\begin{enumerate}
    \item $d(x,y)\geq 0$
    \item $d(x,y)=0$
    \item $d(x,y)=d(y,z)$
    \item $d(x,y) \leq d(x,z)+d(y,z)$
\end{enumerate} 
The part that is different is that, for metric spaces, we require that if the distance is $0$, the points are the same. But for psuedometric spaces, the points can be different yet of the same distance.}
\thmp{Coordinate-wise convergence in $\RR^n$ with $d_p$ metric}{With the metric $d_p(x,y)=(\sum_{i=1}^n (|x_i-y_i|^p))^{1/p}$ on $\RR^n$, convergence implies coordinate wise convergence}{Suppose a sequence $\vec{x_k}$ converges to $\vec{x}$, i.e, $(x_1^{k},x_2^{k} \cdots x_{n}^k)\to (x_1,x_2,\cdots x_n)$. This means that for every $\varepsilon>0$, there exists a $n_0 \in \NN$ so that $\forall k \in \NN$, $k \geq n_0$ would mean that $$(\sum_{i=1}^n(|x_i^{k}-x_i|^p) )^{1/p}<\varepsilon $$ From here, coordinate wise convergence is obvious.
\\\\ Suppose that $\lim_{k \to \infty} x_j^{k}= x_j$ for every $j \leq n$, i.e, coordinate wise convergence. That means that for a given $j$,for every $\varepsilon>0$, there exists $n_0(\varepsilon,j)$ such that $\forall k \in \NN$, $k \geq n_0(\varepsilon,j)$ means that $|x_j^k-x_j|<\frac{\varepsilon}{n^{1/p}}$. If we take the maximum of all the $n_0(\varepsilon,j)$, and sum $( \sum_{i=1}^n|x_{j}^k-x_j|^{p})^{1/p}= \sum_{i=1}^{n}(\frac{\varepsilon^p}{n})^{1/p}=\varepsilon$, which implies convergence in $\R^n$. In a similar fashion, we can show that $\RR^n, d_{\infty}$, the max norm, has equivalence between convergence and coordinate wise convergence  }
\thm{}{$\RR^n$ with the metric $d_{\infty}(x,y)=\max\{|x_j-y_j|: j \leq n\}$ also has an equivalence between convergence and coordinate wise convergence.}
\exm{Cauchy sequences need not converge on $C[0,1]$}{Consider $X=C[0,1]$, with the metric $d(f,g):=\int_{[0,1]} |f(x)-g(x)|dx$. Let $f_n(x)$ be a sequence defined as follows:
$$f_n(x)= \begin{cases}
    0 \ , \text{ if } 0 \leq x \leq \frac{1}{2}-\frac{1}{n} \\
    n(x-\frac{1}{2})+1 \ , \text{ if } \frac{1}{2}-\frac{1}{n}<x \leq \frac{1}{2} \\
    1 \ , \text{ if }\frac{1}{2}<x\leq 1

\end{cases} $$. Is this sequence cauchy? Yes. Consider $\int_{[0,1]}|f_m(x)-f_n(x)|dx \leq \frac{1}{2}(n+m/(mn))$. For any given $\varepsilon$, we can make $m,n$ large enough to go below it. Suppose that $f_n$ converges to some $f$, implying that for every $x \in [0,1]$, we have that $\forall \varepsilon, $ $\exists n_0 \in \NN$ so that $\forall n \in \NN$, $n \geq n_0$ implies $d(f_n,f)=\int_{[0,1]}|f_n(x)-f(x)|<\varepsilon$. $d(f_n,f)=\int_{[0,\frac{1}{2}-\frac{1}{n}]} |f(x)|dx +\int_{[\frac{1}{2}-\frac{1}{n},\frac{1}{2}]}|f_n(x)-f(x)|dx+\int_{[\frac{1}{2},1]}|1-f(x)|dx$ but since $\lim_{n \to \infty}(d(f_n,f))\to 0$, we require that $\int_{[0,\frac{1}{2}-\frac{1}{n}]} |f(x)|dx=0$ and $\int_{[\frac{1}{2},1]}|1-f(x)|dx=0. $ Since $f$ is continuous (or supposed to be), we see that $f(x)=0$ from 0 to $\frac{1}{2}$, but 1 from $\frac{1}{2}$ to $1$, which is absurd. }
\lemp{A useful lemma comparing norms on $\RR^n$}{For $\RR^n$, we have $$|| \cdot ||_p \leq ||\cdot||_1 \leq n ||\cdot||_{\infty} \leq n||\cdot||_{p}$$}{Consider any vector $v$ in $\RR^n$, $v=(x_1,x_2 \cdots ,x_n)$. We have $||v||_p:=(\sum_{i=1}^n|x_i|^p)^{1/p} \leq (\sum_{i=1}^n |x_i|) \leq n \max\{x_i: i\leq n\}\leq n(\sum_{i=1}^p |x_i|^p)^{1/p}$}
\defn{Complete Metric Space}{A metric space $X, d$ is said to be \emph{complete} if every cauchy sequence in $X$ converges in $X$}
\thm{}{$R$ with $d(x,y)=|x-y|$ is a complete metric space}
\thmp{}{$\RR^{n}$ with $d_{p}$, $d_{\infty}$, is complete.}{Consider a cauchy sequence $\{x^{(k)}\}$ in $\RR^n$, i.e, $x^(k)=(x^{(k)}_1,x^{(k)}_2,x^{(k)}_3, \cdots x^{(k)}_n)$ so that $\forall \varepsilon>0$, $\exists m_0(\varepsilon)$ so that $\forall a,b \geq m_0(\varepsilon)$ we have $d(x^{(a)},x^{(b)})<\varepsilon$ i.e:
$$|x_i^{a}-x_i^{b}|\leq (\sum_{i=1}^{n} |x^{(a)}_i-x^{(b)}_i|^p)^{1/p} < \varepsilon$$ which implies coordinate wise cauchy, which means coordinate wise convergence. That is enough to ensure full convergence. This holds for both $d_p$ and $d_{\infty}$.}
\thmp{}{$\ell_p$ space, space of all $p$- summable sequences in $\RR$, is a complete metric space. }{Let $\{x^{k}\}$ be a sequence of sequences where $x^{(k)}=(x^k_1,x^k_2 \cdots)$. The norm here is $d(x^{a},x^{b})=(\sum_{i=1}^{\infty}|x^a_i-x^b_i|^p)^{1/p}$. $\forall \varepsilon>0$, $\exists m_0$ so that $\forall a,b \geq m_0$, $d(x^{a},x^{b})=(\sum_{i=1}^{\infty}|x^a_i-x^b_i|^p)^{1/p}<\varepsilon$. Note that $x^k_i$ is point wise cauchy, hence pointwise convergent to say, $x^{(k)}_i \to x_i$. Define $x:=(x_1,x_2 \cdots)$. For a given $\varepsilon$ we have $n_0$ so that $$(\sum_{i=1}^j|x^a_i-x^b_i|^p )^{1/p}\leq(\sum_{i=1}^{\infty}|x^a_i-x^b_i|^p )^{1/p}<\varepsilon  $$ for all $a,b \geq n_0$, (for every $j$, too). We take the $b \to \infty$ limit here, to get: $$(\sum_{i=1}^j|x^a_i-x_i|^p )^{1/p}<\varepsilon $$ for all $a\geq n_0$ and $\forall j$. From monotone convergence theorem, we can take the $j$ limit as well, to arrive at: $\forall \varepsilon$, $\exists n_0$ so that $\forall a \geq n_0$, we have $$d_p(x^a,x)(\sum_{i=1}^{\infty} |x^a_i-x_i|^p)^{1/p}<\varepsilon $$ which lets us conclude $x^k \to x$. But the question is, is $x \in \ell_p$? Consider $(\sum_{i=1}^{j}|x_i|^p)^{1/p} \leq (\sum_{i=1}^j|x^a_i-x_i|^p )^{1/p}+(\sum_{i=1}^j|x^a_i|^p )^{1/p} $ (Minkowski). Both the summands on the right side are convergent (by MCT, and definition, respectively). Hence $x \in \ell_p$, which makes $\ell_p$ a complete space}
\thmp{}{Set of all bounded sequences on $\RR$ with $d_{\infty}$ norm is complete}{ Consider $\{x^{k}\}$ where $x^{k}=(x^k_1,x^k_2 \cdots)$, a sequence that is cauchy. $\forall \varepsilon>0$, $\exists n_0$ so that $\forall m,n \geq n_0$, we have $|x^n_i-x^m_i|\leq\sup\{|x^m_i-x^n_i|: i \in \NN\}<\varepsilon$ (for every $i \in \NN$). This ensures coordinate wise convergence. Let $x^n_i \to x_i$, and let $x=(x_1,x_2 \cdots)$ so that $\forall \varepsilon>0$, $\exists m_0$ so that $\forall n \geq m_0$, we have $|x^n_i-x_i|<\varepsilon$ for all $i \in N$. This means for all $n \geq m_0$, $\sup\{|x^n_i-x_i|:i\in \NN\}<\varepsilon$, which is the condition for convergence. Hence, $x^{n} \to x$. Is $x$ a bounded sequence? We require that $|x_i|<M$ for some $M$, for all $i \in \NN$. Fix an $\varepsilon$, and get a corresponding $n$ and fix it so that $|x^n_i-x_i|<\varepsilon$ for all $i \in \NN$, we then have: $|x_i|\leq |x^n_i-x_i|+|x^n_i|\leq \varepsilon+|x^n_i|<$ some fixed number (since $x^n \in $ space of all bounded sequences $B(\NN)$), hence we see that $B(\NN)$ is cauchy complete. }
\thmp{}{Set $C[0,1]$ of all continuous functions on closed, bounded interval $[0,1]$ is cauchy complete under the supremum norm.}{Consider $\{f_n\}$ a sequence of continuous functions on $[0,1]$ that is said to be cauchy. This implies that, $\forall \varepsilon>0$, $\exists n_0$ so that $\forall n,m \geq n_0$, $\sup\{|f_n(x)-f_m(x)|: x \in [0,1]\}<\varepsilon$. This means that $|f_m(x)-f_n(x)|<\varepsilon$ for all $x \in X$ (beyond a certain $n_0$). Hence, $f_m(x)$ converges to say $f(x)$. Moreover, note that for every point $x$, $\forall \varepsilon>0$, $\exists n_0(\varepsilon)$ (independent of $x$) so that $|f_n(x)-f(x)|<\varepsilon$ for $n \geq n_0(\varepsilon)$. It is obvious then that $f(x)$ is bounded on $[0,1]$. We also have that $\forall \varepsilon>0$ $\exists n_0(\varepsilon)$ so that $\sup\{f_n(x)-f(x): x \in [0,1]\}<\varepsilon$, which tells that $f_n \to f$ (uniformly). Now all that is left to see is that $f$ is continuous.
\\\\ Let $\varepsilon>0$ be arbitrary. For every function $f_n$, there exists $\delta(n)(\varepsilon)>0$ so that $\forall a,b \in [0,1]$, $0<|a-b|<\delta \implies |f_n(a)-f_n(b)|<\varepsilon/4$. We also note that, there exists an $n_0$ so that for all $m,n>n_0$, we have for every $x \in [0,1]$, $|f_n(x)-f_m(x)|<\varepsilon/4$ for all $m,n \geq n_0$. Moreover, we also have that there exists $n' \in \NN$ so that $|f_n(x)-f(x)|<\varepsilon/4$ for all $x$, $\forall n \geq n'$.
\\\\ $|f(a)-f(b)|=|f(a)-f_n(a)+f_n(a)-f_m(a)+f_m(a)-f_m(b)+f_m(b)-f(b)|$ $\leq |f_n(a)-f(a)|+|f_m(a)-f_m(b)|+|f_n(a)-f_m(a)|+|f_m(b)-f(b)| $ There exists $\delta$, minimum of the two $\delta_n,\delta_m$ that ensures that the first two modulii go lower than $\varepsilon/4$. There exists large enough $m,n$ to ensure the remaining terms go lower than $\varepsilon/4$ as well. All together, we have that $\exists \delta>0$ so that if $a,b \in [0,1]$ such that $0<|a-b|<\delta$, we have $|f(a)-f(b)|<4(\varepsilon/4)=\varepsilon$. Hence, $f$ is continuous. Therefore, $C[0,1]$ (and more generally, $C[\II]$) is complete. }
\newpage
\subsection{More basics- Closures, Relativeness, Interiors and Boundaries}
\defn{Limit Point of a set $E$}{We say $p$ is a limit point of a set $E$ if $$(\forall \varepsilon>0)(\exists q_{\varepsilon} \in E; q_{\varepsilon} \neq p)(d(q_{\varepsilon},p)<\varepsilon)$$ In other words, in every $\varepsilon$-ball around $p$, there would exist a point $q_{\varepsilon}$ in $E$, which is different from $p$.  }
\thmp{}{Every ball / neighbourhood of $p$ which is a limit point of $E$, would contain infinitely many points $q$ such that $q \in B_{\varepsilon}(p) \cap E\setminus \{p\} $}{Suppose for some neighbourhood, there only exists finite points $q_1,q_2,\cdots q_k$ such that $q_j \in B_{\varepsilon_0} \cup E\setminus\{p\}$. Let $\delta<min\{d(p,q_j): j \in [1,2,...k]\}$. We then have that, there exists no point $q \in E$ such that its distancce from $p$ is less than $\delta$, making $p$ a non-limit point. Absurd.}
\cor{A finite set has no limit points}
\thmp{Recharacterisation of Limit points}{A point $p \in X$ is a limit point of $E \subset X$ if and only if there exists a sequence $x_n \in E$, $x_n\neq p \forall n \in \NN$ such that $\{x_n\} \to p$}{$\implies$) If $p$ is a limit point, around every neighbourhood, there would exist a point $q_\varepsilon \in E$ such that $0<d(q_\varepsilon,p)<\varepsilon$. Choose $\varepsilon_1=1$, and obtain $x_1$ such that $x_1 \in E$, $x_1 \neq p$ and $0<d(x_1,p)<1$. Choose $\varepsilon_2=\frac{1}{2}(d(x_1,p))$. We find $x_2 \in E$, $x_2 \neq p$ such that $d(x_2,p)<\frac{d(x_1,p)}{2}<\frac{1}{2}$. Continue as such to obtain a sequence that converges to $p$.
\\\\ $\impliedby$) Suppose there is a sequence $x_n$ such that $x_n \neq p \forall n \in \NN$ and $\forall \varepsilon, \exists n_0(\varepsilon)$ such that $\forall n \geq n_0$ we have $d(x_n,p)<\varepsilon$ which means for a given $\varepsilon$, there exists a point $x_{n_0+1}$ in $E$ such that it is not equal to $p$ and it is in the $\varepsilon$-ball of $p$. Hence, $p$ would be a limit point. }
\defn{Closed sets in $X$}{A set $E$ is closed in $X$ if every limit point of $E$ is contained in $E$}
\defn{Equivalent definition of closed sets in $X$}{A set $E$ in $X$ is closed if for every convergent sequence $x_n$ in $x$ such that $lim(x_n) \neq x_n$ for any $n$, we have $lim(x_n) \in E$.}
\defn{Open sets in $X$}{A set $E$ is said to be open if $\forall x \in E$, $\exists \xi_x>0$ such that $B_{\xi_x}(x) \subset E$}
\thmp{}{Every open ball is an open set}{Suppose $a$ is a fixed point in $X$ and $\delta>0$ is given. $B=B_{\delta}(a):=\{y \in X: d(y,a)<\delta\}$.Consider arbitrary $z \in B$, for which we have $d(z,a)=t<\delta$. Therefore $\delta-t>0$. Consider $0<\xi_z=r<\delta-t$ from Density. Consider an arbitrary $x$ such that $d(x,z)<\xi_z=r<\delta -t$. $d(x,a)\leq d(x,z)+d(a,z)=r+t\leq \delta -t+t=\delta$. We are done.}
\defn{Compliment with respect to $X$}{If $E \subseteq X$, we define compliment of $E$ as $$E^C:=\{ x\in X:x \not\in E\}$$}
\defn{Bounded}{A set $E \subset X$ is bounded if $\exists$ a positive number $M>0$ and $q \in E$ such that $d(x,q)<M$ $\forall x \in E$. i.e, all the points of $E$ gets contained in some ball in $X$. }
\thmp{De Morgan's Law}{Let $\{E_{\alpha}: \alpha \in A\}$ where $A$ is some arbitrary indexing set represent a collection of sets in $X$. Then $$(\Bigg.\cup_{\alpha} E_{\alpha})^{C}=\Bigg.\cap_{\alpha}E^C_{\alpha}$$}{Consider $(\cup_{\alpha} E_{\alpha})^c=\{x \in X: \exists \alpha \in A: x \in E_{\alpha} \}^c=\{x \in X: \forall \alpha \in A: x \not\in E_{\alpha} \}=\{x \in X: \forall \alpha \in A: x \in E^c_{\alpha} \}=\cap_{\alpha}E^c_{\alpha}$}
\thmp{The Big Equivalence}{$E \subset X$ is open $\iff$ $E^c$ is closed.}{
$\implies$) Suppose that $E$ is open but $E^c$ is not closed. This means that there exists a limit point of $E^c$ that falls in $E$, i.e, outside $E^c$. Let this be $q$. This means for every $\varepsilon$-ball around $q$, a point of $E^c$ exists. But since $E$ is open and $q\in E$, we have for a particular $\varepsilon$-ball, inside which, no point of $E^c$ resides. Contradiction.\\\\
$\impliedby$) Suppose $E$ is closed but $E^c$ is not open. This means that there is a point in $E^c$, $p$, such that for every $\varepsilon$-ball around $p$, some point in $E$ falls into this ball. But this makes $p$ a limit point of $E$, which is absurd since $E$ is closed, limit points fall into the sets themselves.
}
\thmp{}{For a collection of open sets $\{G_{\alpha} :\alpha \in A\}$, $\cup_{\alpha} G_{\alpha}$ is also an open set.}{Consider $x \in \cup_{\alpha} G_{\alpha}$ which means $\exists \alpha_x \in A$ such that $x \in G_{\alpha_x}$ which means, there would exist an $\xi$-ball around $x$ that is contained in $G_{\alpha_x}$ which is in turn contained in $\cup_{\alpha} G_{\alpha}$.}

\corp{For any collection of closed sets $E_{\alpha}$, $\cap_{\alpha} E_{\alpha}$ is also closed.}{$\{ E^c_{\alpha}\}$ is a collection of open sets, and $\cup_{\alpha} E^c_{\alpha}$ is an open set, which means $\cup_{\alpha}E^c_{\alpha}=(\cap_{\alpha} E_{\alpha})^c$ is an open set, from which we get that $(\cap_{\alpha} E_{\alpha})$ is a closed set.}

\thmp{}{For any finite collection of open sets $\{E_1,E_2,\cdots E_k\}$, $\cap_{i=1}^kE_i$ is also open.}{Suppose $x \in \cap_{j=1}^kE_j$, which means $\forall j\in[1,k], x \in E_j$. We have $\varepsilon_1,\varepsilon_2,\cdots \varepsilon_k$ such that, the $\varepsilon_j$-ball around $x$ is fully contained in $E_j$. Choose $0<\delta<min\{\varepsilon_1,\varepsilon_2,\cdots,\varepsilon_\}$ (the minimum exists by virtue of being a finite set). We see that the $\delta$-ball around $x$ is a subset of every $\varepsilon_j$-ball around $x$, which means that the $\delta$-ball around $x$ is in every $E_j$, which proves the theorem.}
\cor{For any finite collection of closed sets $\{G_1,G_2,\cdots G_k\}$ we have $\cup_{j=1}^k G_j$ to be closed}
\rmkb{
In the above theorem and corollary, we require that the collection be finite. The reason is that, we were able to get a minimal $\varepsilon_j$ in the proof due to the finiteness of the set. It may not be possible to find a number $\delta$ that is both larger than $0$ but smaller than a given infinite collection of $\varepsilon$-s. For example, consider the sequence of open sets $(-\frac{1}{n},\frac{1}{n})$. The infinite intersection of these yields $\{0\}$ which is a closed set by virtue of being finite.
}
\defn{Topology}{Let $X$ be a set and $\tau $ be a family of sets in $X$. $X$,$\tau$ is called a topology (with elements of $\tau$ being called open sets) if:
\begin{enumerate}
\item Both $\phi$ and $X$ are in $\tau$
\item $\tau$ is closed under arbitrary unions
\item $\tau$ is closed under finite intersections
\end{enumerate}
The complement of an element in $\tau$ with respect to $X$ is called a closed set.
}
\defn{Closure of a set}{Let $E'$ be the set of all limit points of $E$. Then, the closure of $E$ is : $$\bar{E}:=E \cup E' $$}
\thmp{}{Closure of a set is closed}{Let $p$ be a limit point of $E \cup E'$. That means that $\forall \varepsilon>0$ $\exists q_{\varepsilon} \in (E\cup E'), q_{\varepsilon} \neq p$ such that $q_{\varepsilon} \in B_{\varepsilon}(p)$. If $p$ is in $E \cup E'$, we are done (especially if $p$ is in $E$). Suppose $p$ is not in $E$. $\forall \varepsilon>0$ $\exists q_{\varepsilon} \in (E\cup E'), q_{\varepsilon} \neq p$ such that $q_{\varepsilon} \in B_{\varepsilon}(p)$. If the $q_{\varepsilon}$ we recieve falls in $E$ we are ok. Suppose $q_{\varepsilon}$ falls in $E'$. That means: $\forall \delta>0$, $\exists r_{\delta} \in E$, $r \neq q_{\varepsilon}$ such that $d(r_{\delta},q_{\varepsilon})<\delta$ $\implies$ $d(r_{\delta},p) \leq d(r_{\delta},q_{\varepsilon})+d(q_{\varepsilon},p)<\delta+d(q_{\varepsilon},p)<\delta+\varepsilon $ If we choose $\delta_0<{\varepsilon-d(q_{\varepsilon},p)}$ we get:
$d(r_{\delta},p) \leq d(r_{\delta},q_{\varepsilon})+d(q_{\varepsilon},p)<\delta+d(q_{\varepsilon},p)<\varepsilon$
\\\\Summarising we have: $\forall \varepsilon>0$, $\exists q_{\varepsilon} \in E$ or $E'$ where: $q_{\varepsilon} \in E$ and $q_{\varepsilon} \in B_{\varepsilon}(p)$ \\\\ or \\\\ $\exists \delta(\varepsilon)>0$ such that $\exists r_{\delta} \in E$ such that $r_{\delta} \neq p$ and $r_{\delta} \in B_{\varepsilon}(p)$. In either case, there would exist a point dependent on $\varepsilon$, in $E$ such that the point itself is different from $p$, and exists in the $\varepsilon$-ball around $p$. Hence, we see that $p$ is a limit point of $E$. Therefore, we see that all the limit points of $E$ either are points of $E$ or points of $E'$. Hence, $\bar{E}$ is closed.}
\thmp{}{$\bar{E}=E \iff E$ is closed}{$\implies$)$\bar{E}$ is closed, so $E$ would be too.\\\\
$\impliedby$) if $E$ is closed, $E' \subseteq E \implies E' \cup E = E=\bar{E}$}

\thmp{$\bar{E}$ is the smallest closed set that contains $E$}{If $F_{\alpha}$ is the collection of all closed sets such that $E \subseteq F_{\alpha}$, then $\bar{E} \subseteq F_{\alpha}$ for all $\alpha$.}{Consider an arbitrary closed set $F_{\alpha}$ that contains $E$. It would obviously contain all the limit points of $E$ among other things. Therefore, we can easily see that it contains $E\cup E'=\bar{E}$.}
\fact{\emph{Topological Definition for Closure}: The closure of $A \subseteq X$, denoted by $\overline{A}$ is defined as the smallest closed set that contains $A$}
\lemp{An equivalent definition for Closure.}{An equivalent definition for closure is: $$\bar{A}:=\{x \in X: \forall \varepsilon>0, B_{\varepsilon}(x) \cap A \neq \phi\}$$}{We see that obviously, if $x \in \bar{A}$, then either it is a point of $A$, or if not, it happens to be a limit point of $A$. And the back implification: If $q$ is a point of $A$ or if it is a limit point o $A$, it obviously falls into $\bar{A}$.}

\exm{If $E \subseteq \RR$ is bounded (and non empty), with $s=sup(E)$, then $s \in \bar{E}$}{If $s \in E$ we are done. If not, then $\forall \varepsilon>0$, $\exists \varepsilon>\delta(\varepsilon)>0,$ and a point $x_{\varepsilon} \in E$ such that $s-\varepsilon<s-\delta(\varepsilon) \leq x_{\varepsilon}<s+\delta<s+\varepsilon$ where $x_{\varepsilon}\neq s$. Hence, $s$ is a limit point of $E$ and hence, is a point in the closure.}

\defn{Open Relative}{Say $E \subseteq Y \subseteq X$, where $X$ is a metric space. $Y$ is also a metric space. We say $E$ is open relative to $Y$ if $\forall x\in E$, $\exists \varepsilon>0$ such that if $y \in Y$ and $y \in B_{\varepsilon}(x)$ then $y\in E$. Formally:
$$(\forall x \in E)(\exists \varepsilon_x>0)( (y \in Y \cap B_{\varepsilon}(x))\implies y \in E) $$}
\rmkb{A set which is open relative to $Y$ need not be open relative to $X$. For example, consider $\RR$ as a subset of $\RR^n$. An interval in $\RR$ is open relative to $\RR$, but it is not open relative to $\RR^n$.}
\defn{Limit point relative}{Say $S \subseteq Y \subseteq X$. $x$ is said to be a limit point of $S$ relative to $Y$ if for every $\varepsilon$-ball around $x$ relative to $Y$ (i.e, the set $B_{\varepsilon}(x)\cap Y$, or $\{y \in Y: d(y,x)<\varepsilon\}$) there exists a point $y \in S$. In technical terms: $x$ is a limit point of $$(\forall \varepsilon>0)(\exists y \in Y \cup B_{\varepsilon}(x))(y \in S) $$ }
\defn{Closed relative}{A set $S \subset Y \subset X$ is closed relative to $Y$ if every limit point of $S$ relative to $Y$ is in $S$.}
\rmkb{Closed relative to $Y$ needn't imply closed relative to $X$. For example, a convergent sequence of $S$ that converges in $X$ needn't converge in $Y$. As a result, if this sequence that converges in $X$ (but not in $Y$) converges outside $S$, under the subspace topology of $Y$, this simply does not contribute as a limit point. Whereas, in $X$ this contributes as a limit point, and hence is not closed in $X$.}
\rmkb{Note that, we needn't define the concept of "relative limit point" since every limit point relative to $Y$, is a limit point of $S$ relative to $X$. But the back direction needn't be true. i.e, you can have $x$ a limit point of $S$ relative to $X$, but not relative to $Y$. More concretely, $x$ is such that for every $\varepsilon>0$, there exists $z \in S$ so that $z \neq x $ and $d(z,x)<\varepsilon$. If $z$ is in $X$ as well as $Y$, its a limit point of $S$ with respect to both. Issue arises if $x$ is not in $Y$. Then it does not make sense. }
\thmp{}{A set $E\subseteq Y \subseteq X$ is open relative to $Y$ $\iff$ $\exists G \subset X$ that is open relative to $X$, such that $E=G \cap Y$ }{$\implies$) Say $E$ is open relative to $Y$. This means that $\forall x \in E$, $\exists \varepsilon_x>0$ such that if $y \in B_{\varepsilon_x}(x)$ and $y \in Y$, then $y \in E$. Call $G=\cup_{x\in E}B_{\varepsilon_x}(x)$ which is an open set. If $z \in E$, then $z \in G$ obviously, and hence $z \in G \cap Y$. Hence, $E \subseteq G \cap Y$. Consider a point $z \in G \cap Y$ which means $z$ falls in one of the $\varepsilon$-balls around a point of $E$, and $z$ is in $Y$. From definition of open relativeness, we see that $z \in E$. Hence, $E=G \cap Y$
\\\\ $\impliedby$) Say $E=G \cap Y$ where $G$ is an open set relative to $X$. Then, for every point in $G$, there would exist an $\varepsilon$-ball around that point that is completely contained in $G$. Let $x\in E$ be arbitrary. $\exists \varepsilon_x>0$ such that $B_{\varepsilon}(x) \subset G$. Suppose $y \in Y$ and $y \in B_{\varepsilon}(x)$. This would mean that $y \in G\cap Y=E$. Hence, $\forall x\in E$ $\exists \varepsilon>0$ such that if $y\in B_{\varepsilon}(x)$ and $y\in Y$, then $y\in E$, which is the definition of open relativeness.}
\fact{\emph{Topological definition of "open-relative"}: Let $A \subseteq Y \subseteq X$. $A$ is said to be open relative to $Y$ if there exists an open set $G$, open in $X$ so that $A=G \cap Y$.}
\thmp{}{A set $S$ is open relative to $Y \subseteq X$ if and only if $Y\setminus{S}$ is closed relative to $Y$}{Once we notice that replacing $B_{\varepsilon}(x) \cap Y$ with $B^Y_{\varepsilon}(x)$, i.e, the corresponding $\varepsilon$ ball of $x$ with respect to $Y$, the proof falls in much the same way as the earlier version, done for $X$. }
\thmp{}{$S$ is closed relative to $Y \subseteq X$ if and only if $S=H \cap Y$ for a closed set $H$ closed in $X$}{Say $S$ is closed, that means $Y \setminus S=K$ is open relative to $Y$. That means there exists $G$ open relative to $X$ so that $G \cap Y=K$. $Y\setminus(K)=Y \cap K^C=Y \cap (G \cap Y)^C=Y \cap (G^C \cup Y^C)=(Y\cap G^C)\cap (Y \cup Y^C)=Y\cap G^c$. We have that $Y\setminus K=Y \setminus(Y\setminus S)=S=Y \cap G^C$. 
\\\\ Let $S=H \cap Y$ where $H$ is closed in $X$. $Y\setminus S=Y \setminus(H \cap Y)=Y \cap(H^C \cup Y^C)=Y\cap H^C$, we done.}
\thmp{}{Suppose $A_1, A_2 \cdots, A_n \cdots \in X $. Then,
\begin{enumerate}
\item If $B_n= \cup_{i=1}^n A_i$, then $\bar{B_n}=\cup_{i=1}^{n} \bar{A_i}$ 
\item $B=\cup_{i=1}^{\infty} A_i$, $\bar{B} \supseteq \cup_{i=1}^{\infty}\bar{A_i}$ with possibility of strict inequality.
\end{enumerate} and as a result we have for open sets $G_i$:
\begin{enumerate}\item ${(\cap_{i=1}^n G_i)}^{\circ}=\cap_{i=1}^n(G_i)^{\circ}$  \end{enumerate} }{Suppose $x \in \cup_{i=1}^{n}\bar{A_i}$. Which means
$x \in \bar{A_i}$ for some $i$. It is clear that $x$ is either a 
point of $A_i$ or a limit point of $A_i$. We have that, whatever maybe the case
either $x$ is a point in $B_n$ or a limit point of $B_n$. Therfore
$\bar{B_n}\supset \cup_{i=1}^{n} \bar{A_i}$. 
\\\\ Suppose $x$ is a point in $\bar{B_n}$. If it is a point of $B_n$,
we are done. Suppose it is the limit point of $B_n$, but not a point. Also
suppose that $x$ is not a limit point of any $A_i: i=1 \to n$. This means that,
$$\exists \varepsilon_1 \text{ such that } \forall q\in A_1, q\neq x, \text{ we have } d(q,x)\geq \varepsilon_1$$
$$\exists \varepsilon_2 \text{ such that } \forall q\in A_2, q\neq x, \text{ we have } d(q,x)\geq \varepsilon_2$$
$$\vdots$$
$$\exists \varepsilon_n \text{ such that } \forall q\in A_n, q\neq x, \text{ we have } d(q,x)\geq \varepsilon_n$$
If we choose $0<\varepsilon_0<min\{\varepsilon_i: i=1 \to n \}$, we would have that,
for every point $q$ in $A_1 \cup A_2 \cdots A_n$, $q \neq x $(which is needless to say), we have $d(q,x) \geq \varepsilon_0$ which 
makes $x$ a non-limit point of $B_n$, which is absurd. Hence we see that if $x$ is point of $B_n$ or a limit point of $B_n$, then it is a point or the limit point of some $A_j$.
\\\\ For a good counterexample, we look to $\QQ:=\{q: \text{ q is rational} \}$. This set is 
countable. Let $\{q_1, q_2, \cdots \}$ be the enumeration of $\QQ$. Consider $\QQ:=\cup_{j=1}^n \{q_j\}$. The closure
of $\QQ$ is $\RR$ but since these singleton sets are by definition closed, the union of them only gives you $\QQ$.
\\\\ \textbf{Proof for topological space}:
\\ if $B_n:=\cup_{i=1}^{n}A_i$, is it true in topological spaces (and not just metric spaces) that $\overline{B_n}=\cup_{i=1}^n \overline{A_i}$? the direction $\overline{\cup_{i=1}^n A_i}\subseteq \cup_{i=1}^n \overline{A_i}$ can be shown from definition as the smallest closed set that contains $B_n$.  But $\overline{A_i} \subseteq \overline{\cup_{i=1}^n A_i}$ for every $i$, so if we union them we must get $\cup_{i=1}^n \overline{A_i}\subseteq \overline{\cup_{i=1}^n A_i }$}

\defn{Interior of a Set}{Given $S \in X$, the interior $\underbar{S}$ is defined as:
$$\underbar{S}:=\{x\in X: \exists \varepsilon_x>0 \text{ such that } B_{\varepsilon_x}(x) \subset S \} $$}
\thmp{}{The Interior is an open set}{Consider $(\underbar{S})^C:=\{x \in X: \forall \varepsilon>0, B_{\varepsilon}(x) \cap S^C \neq \phi \}$.
It is possible that $x$ is a point of $S^C$, if it is in $\underbar{S}^C$. Suppose
it is a point of $\underbar{S}^C$ but not a point of $^C$. From the definition, we see
that $\forall \varepsilon>0, \exists q \in S^C, q\neq x$ such that $d(q,x)<\varepsilon$.
This makes $x$ a limit point of $S^C$, which means, for every $x$ in $\underbar{S}^C$,
$x$ is either a point of $S^C$ or a limit point of $S^C$. Hence,
$$\underbar{S}^C \subseteq \bar{S^C} $$
Suppose $x$ is a point of $\bar{S^C}$. Say it is a point of $S^C$, then obviously,
it is a point of $(\underbar{S})^C$. Suppose $x$ is not a point of $S^C$, but a limit
point of $S^C$. This means $\forall \varepsilon>0, \exists q \in S^C, q \neq x$ so that
$d(q,x)<\varepsilon$. This is precisely the condition for which $x$ is a point of $(\underbar{S})^C$.
Hence we see $(\underbar{S})^C \supseteq \bar{(S^C)} $. Therefore, $(\underbar{S})^C=\bar{(S^C)}$. From here we 
see that $\underbar{S}$ is an open set.
\\\\ \textbf{Alternate Argument (similar):} Consider $(\underline{S})^C:=\{x \in X: \forall \varepsilon>0, B_{\varepsilon}(x)\cap S^C \neq \phi \}$
Consider a limit point $p$ of $(\underbar{S})^C$. $\forall \varepsilon>0, \exists q_{\varepsilon} \in \underbar{S}^C$ such that $d(q_{\varepsilon},p)<\frac{\varepsilon}{2} $. 
Since $q_{\varepsilon}$ is in $(\underline{S})^C$, we have that: $\forall \delta>0, \exists r_{\delta} \in S^C, r_{\delta}\neq q_{\varepsilon}$ such that $d(r_{\delta},q_{\varepsilon})<\delta$

Combining these we have:
$$(\forall \varepsilon>0)(\exists q_{\varepsilon} \in \underbar{S}^C)(\exists \delta>0)(\exists q_{\delta} \in S^C)$$
$$(d(q_{\delta},p)\leq d(q_{\delta},q_{\varepsilon})+d(q_{\varepsilon},p)<\frac{\varepsilon}{2}+\delta<\varepsilon) $$
This means $p$ is a limit point of $S^C$. Hence, $\underbar{S}^C$ is closed.}
\thmp{}{$\underbar{S}=S$ $\iff$ $S$ is open}{$\implies $) if $\underbar{S}=S$, obviously 
$S$ is open.
\\\\ $\impliedby)$ If $S$ is open, then by definition $\underbar{S}=$ set of all
points in $S$ so that there's an $\varepsilon$-ball of $x$ in $S$. But that is every point
of $S$. }
\thmp{}{$\underbar{S}$ is the largest open set contained in $S$}{Consider an open
subset of $S$. These are subsets of $S$ whose each point has an $\varepsilon$-ball
around it so that the ball is contained in the subset, which is contained in $S$. So by 
definition, these points in these subsets are contained in $\underbar{S}$.}
\thmp{}{$$(\underbar{S})^C=\overline{(S^C)}$$ \centering{"The compliment of the interior is the closure of the compliment"}}{Refer to the proof of "Interiors of sets are Open", to see this construction.
\\\\ \textbf{Alternate method(slicker):}
\\\\ $\underbar{S} \subseteq S\implies S^C \subset (\underbar{S})^C$ where $(\underbar{S})^C$
is a closed set containing $S^C$. Since $\overline{S^C}$ is the smallest closed set that contains $S^C$, we
have $\overline{S^C} \subseteq (\underbar{S})^C $.
\\\\ Note that $S^C \subseteq \overline{S^C} \implies (\overline{S^C})^C \subseteq S$ where $(\overline{S^C})^C$
is an open set inside $S$. Since $\underline{S}$ is the largest open set containing $S$,
we have that $ (\overline{S^C})^C \subseteq \underline{S} \implies \underline{S}^C \subseteq \overline{S^C}$. Combining these
two set inequalities, we are done.
}

\newpage
\defn{Isometry}{$(X,d_X)$ is said to be isometric to $(Y,d_Y)$ if there exists a function $\phi:X \to Y$ reffered to as the isomoetry, such that $d_Y(\phi(x),\phi(y))=d_X(x,y)$ for all $x,y \in X$}
\thmp{}{Let $X,d_X$ and $Y,d_Y$ be isometric under the map $\varphi:X \to Y$. Then the map $\varphi$ is injective, and its inverse $\varphi^{-1}:\phi(X)\to X$ is also also an isometry. }{Suppose $\varphi(x)=\varphi(y)$. Then $d_Y(\varphi(x),\varphi(y))=d_X(x,y)=0 \implies x=y$. 

$d_X(\varphi^{-1}(\varphi(a)),\varphi^{-1}(\varphi(a)))=d_X(a,b)=d_Y(\varphi(a),\varphi(b))$ which makes $\varphi^{-1}$ an isometry from $\varphi(X) \to X$.}
\thmp{}{Let $Z$ be a set and $f:Z \to (X,d)$ be any injective map from $Z$ to $X$. Then $\delta(x,y):=d(f(x),f(y))$ would form a metric on $Z$ and make it a metric space, and this would in turn make $f$ an isometry from $(Z,\delta)$ to $(X,d)$}{$\delta(x,y)=d(f(x),f(y))$ would obey all the properties of a metric space, and moreover, $\delta(x,y)=d(f(x),f(y))$ making it by definition an isometry. }
\defn{Diam}{$Diam(A \subseteq X):= \sup\{d(x,y):x \in X, y \in X\}$}
\defn{Distance between a point and a set}{$d(x,A):=\inf\{d(x,a):a \in A\}$}
\thmp{}{$x \in \overline{A} \iff d(x,A)=0$}{Obviously, if $d(x,A)=0$, either $x \in A$ or $\forall \varepsilon>0$, $\exists a_{\varepsilon} \in A, a_\varepsilon \neq x$ such that $d(x,a_{\varepsilon})<\varepsilon$, making $x$ a limit point of $A$. The other direction is just as trivial. }
\defn{Isolated point}{$x \in X $ is said to be an isolated point of $S \subseteq X$ if $x \in S $ and $d(x,S\setminus{\{x\}})\neq 0$ or rather, there is a ball around $x$ such that $B_{\varepsilon}(x) \setminus\{x\}$ is completely disjoint from $S$.}
\rmkb{In some sense, being an isolated point of a set is the opposite of being a limit point. $x$ is a limit point of $S$ if for every $\varepsilon>0$, there exists a point $s_{\varepsilon} \in S \setminus\{x\}$ so that $s_{\varepsilon} \in B_{\varepsilon} \setminus\{x\}$ or rather for every $\varepsilon>0$, $B_{\varepsilon}(x) \setminus\{x\} \cap S \neq \emptyset$. The negation of this statement is that there exists an $\varepsilon>0$ so that $B_{\varepsilon}(x) \setminus\{x\} \cap S = \emptyset$ }
\defn{Nearest Point}{Let $S \subseteq X$ and $z \in X$. If there exists $s \in S $ so that $dist(z,S)=d(z,s)$ then $s$ is said to be the \emph{nearest point} of $S$ to $z$.}
\defn{Boundary Point}{A point $x$ in $X$ is said to be a boundary point of $S \subseteq X$ if $dist(x,S)=dist(x,S^c)=0$. Right from the definition, a point $x$ is a boundary point of $S$ if and only if its a boundary point of $S^C$. The \textbf{\emph{Boundary}} of a set is defined as the set of all boundary points of $S$ denoted $\partial(S)$.}
\thmp{}{$\partial(S)=\overline{S}\setminus{S^{\circ}}$}{$\partial{S}$ was defined as the set of all points $x$ so that $dist(S,x)=dist(x,S^C)=0$. By virtue of that, $x \in \partial{S}$ implies $x \in \overline{A}$. Can $x \in S^{\circ}$? That would mean that there is an epsilon ball around $x \in S^{\circ}$ which means that the distance between $x$ and $S^c$ would be non zero. Hence, $\partial{S} \subseteq \overline{S} \setminus\{S^{\circ}\}$ 
\\\\ Now let $x \in \overline{S} \setminus S^{\circ}$. Clear that $dist(x,S)=0$ Obviously. $\overline{S} \cap (S^{\circ})^C=\overline{S} \cap \overline{(S^C)}$. This tells us that $x$ is in $\overline{S^C}$ as well, hence $dist(x,S^C)=0$, making $x \in \partial(S)$. This concludes the proof that $\partial(S)=\overline{S}\setminus {S^{\circ}}$}
\fact{Boundary of a set is closed ($\partial(S)=\overline{S} \cap \overline{S^C}$).}
\subsection{Relative closures, interiors and boundary}
\defn{Relative closure ($cls_Z(S)$)}{The closure of $S \subset Z \subset X$ relative to $Z$, denoted as $cls_Z(S)$ is defined either as the set $S$ along with the set of all limit points of $S$ in $Z$, or equivalently as the smallest closed set, closed relative to $Z$, that contains $S$.}
\defn{Relative Interior ($int_Z(S)$)}{The interior of $S \subset Z \subset X$ relative to $Z$, denoted as $int_Z(S)$ is defined either as the set of all points $z \in Z$ so that there exists an $\varepsilon>0$ ball so that $B_{\varepsilon}(x) \cap Z$ is fully contained in $S$, or as the largest open set, open relative to $Z$, that is inside $S$.}
\thmp{}{Let $S \subseteq Z \subseteq X$. Then, $$cls_Z(S)=cls_X(S) \cap Z $$}{Note that $cls_X(S) \cap Z$ is a closed set, closed relative to $Z$ that contains $S$. Hence, $cls_Z(S) \subseteq cls_X(S) \cap Z$. $cls_Z(S)$ is the smallest closed set relative to $Z$ that contains $S$, which means $cls_Z(S)=H \cap Z$ for some closed set $H$. Note that this closed set must too contain $S$. Hence, we have $cls_X(S) \subseteq H$ which gives $cls_X(S) \cap Z \subseteq H \cap Z= cls_Z(S)$. Hence we have $cls_Z(S)=cls_X(S) \cap Z$.}
\thmp{}{Let $S \subseteq Y \subseteq X$. Then $int_Y(S)=int_X(S \cup (X \setminus Y)) \cap Y$}{$int_Y(S)=\cup_{\substack{G \text{ open} \\ G\cap Y \subset S}} (G \cap Y)=(\cup_{\substack{G \text{ open} \\ G\cap Y \subset S}} G)\cap Y$ which gives $int_Y(S)=(\cup_{\substack{G \text{ open} \\ G \subset S \cup Y^C}} G)\cap Y=int_X(S \cup Y^C) \cap Y$ }

\subsection{Denseness, topologies and more on open / closed sets}
\defn{Dense sets}{A set $A \subseteq X$ is said to be dense in $X$ if $\overline{A}=X$}
\thmp{}{A set $A$ is dense in $X$ if and only if for every open set $G \in X$, $A \cap G \neq \emptyset$}{$\implies$) Suppose $G_0 \cap A =\emptyset$. That means $A \subseteq G_0^C \neq X$ which means, since $G_0^C$ is open, we have $\overline{A}\subseteq G_0^C \neq X$.
\\\\ $\impliedby$) if $\overline{A} \subset X$ properly, then $(\overline{A})^C$ is an open set disjoint from $A$. }


\textbf{Question:} Suppose metric $\gamma: X \times X \to \RR^+ $ induces a topology $\mathcal{U}$. Can the topology tell us the metric? i.e, can we get back $\gamma: X \times X \to \RR^+$ from the topology $\mathcal{U}$? 
\thmp{}{If $\mathcal{U}$ is a topology arising from $\delta(x,y)$, then the same topology can also be achieved by the metric $d(x,y)=\delta(x,y)/(1+\delta(x,y))$, as well as $e(x,y)=\delta(x,y)/2$}{Let $S$ be an open $r$ ball in the $\delta$ metric. $S_r:=\{x \in X: \delta(x,y)<r\}$. Let $T_s$ be the open $s$ ball in the $d$ metric. $T:=\{x \in X: \delta(x,y)/(1+\delta(x,y))<s\}$. Now let $E$ be the $a$ ball in the $e$ metric. $E_a:=\{x \in X: \delta(x,y)/2<a\}$. Easy to see that $E_{r/2}=S_r$ and $E_r=S_{2r}$. Moreover, $T_{r/1+r}=S_r$ and $T_{r}=S_{r/1-r}$}
The previous theorem tells us that the answer to the prior question is: no. Different metrics can give rise to the same topologies. 
\subsubsection{Topologies on Subspaces and Superspaces}
Given a topology of $X$, we would like to classify, or have the structure of the topology generated in a subspace $Y \subset X$. 
\thmp{}{If $\mathcal{U}$ is the topology of $X$, then $\mathcal{U} \cap Y$, that is $\{U \cap Y: U \in \mathcal{U}\}$ is the topology of $Y.$}{We know by definition (and in metric spaces, by virtue of a theorem) that $S \subseteq Y \subseteq X$ is open relative to $Y$ if and only if $S=G \cap Y$ for some open set $G$ relative to $X$. Let $U$ be an open set in $X$. That means that $U \cap Y$ is open relative to $Y$, making it part of the topology. Let $K$ be an open set relative to $Y$. That means $K= G \cap Y$ for some $G$ open relative to $X$. Hence, the topology of $Y$ is such that every element in it is of the form $G \cap Y$ for open $G$ in $X$, and for every open $G \subseteq X$, $G \cap Y$ is part of the topology of $Y$. Hence, $T=\{U \cap Y: U \in \mathcal{U}\}$ }
\thmp{}{Suppose $Z$ is a metric space and $Z \subseteq X$. Topology of $X$ includes that of $Z$ if and only if $Z$ is open in $X$.
\\\\ This essentially means that every set open relative to $Z$ is open relative to $X$ if and only if $Z$ itself is open in $X$. }{$\impliedby)$ Suppose $Z$ is open in $X$. That means $Z$ is part of the topology of $X$. Moreover, If $S \subseteq Z \subseteq X$ is open relative to $Z$ (i.e, is part of the topology of $Z$) then it can be written as $ G \cap Z$ which is also again an open set in $X$. Hence, topology of $Z$ is contained in the topology of $X$.
\\\\ $\implies)$ Suppose the topology of $Z$ is contained in $X$, that means that every open set relative to $Z$ is actually an open set relative to $X$. That makes $Z$ an open set relative to $X$. We are done.}
\textbf{Question:} Are there non empty metric spaces such that they are open relative to every superspace? 
\\\\ \textbf{Answer:} No. Consider $X$ to be a non empty metric space with distance $d$. Consider $X \times \{0\}=Y$. It is obvious that $Y$ is bijective to $X$. Consider $f:X \to Y$ given by $f(x)=(x,0)$. Under the obvious metric $k[(x,0),(y,0)]=k[f(x),f(y)]=d(x,y)$, we can clearly see that $Y,k$ forms a metric space isometric to $X,d$. We can therefore identify $X,d$ with $Y,k$. Now look at $Z=X \times [0,1]$ with the metric $\delta[(x,r_x),(y,r_y)]:=d(x,y)+|r_x-r_y|$. Consider $dist_Z[(x_0,0),Z \setminus Y]$. This is $dist_Z[(x_0,0),X \times (0,1]]$ which is $0$. Hence, $Y$ is not open wrt $Z$. Its over.
\defn{Complete Metric Space}{A metric space is said to be complete if $X$ is closed in every metric superspace}
\lemp{Cantor's Intersection Lemma}{$X$ is a cauchy complete metric space if and only if for every sequence of nested, non empty closed sets $F_n \subseteq F_{n-1} \subseteq F_{n-2}\cdots$ such that $diam(F_n) \to 0$, $\cap_{n \in \NN} F_n$ is a singleton. }{$\implies$) Let $X$ be cauchy complete. Consider $x_1 \in F_1$, $x_2 \in F_2 \cdots x_n \in F_n$ and so on. We are told $diam(F_n) \to 0$, i.e $\forall \varepsilon>0, \exists n_0$ such that $\forall n \geq n_0$, $diam(F_n)<\varepsilon$. This means that $\forall n,m \geq n_0$, we have $d(x_m,x_n)<\varepsilon$. Hence, $(x_n)$ is a cauchy sequence. It converges to $x$. Note that for any $F_k$, only finite points of $(x_n)$ lie outside of $F_k$. Suppose $x$ is not in some $F_k$. That means it is in $F_k^C$, an open set. There is a ball around $x$ that is completely inside $F_k^C$. Around this ball, though, only finite points of $x_n$ exists. This contradicts the fact that $x $ is a limit of $(x_n)$. Hence, $x \in F_k$ for all $k \in \NN$. Hence $x \in \cap_{n\in \NN}F_n$. Since $diam(F_n) \to 0$, its obvious that two points cannot exist in the final intersection.
\\\\ $\impliedby)$ Let $(x_n)$ be a cauchy sequence. Note that, if $\{x_n: n \in \NN\}$ has a limit point, say $x$, then choose $\varepsilon_1=1$ and pick $x_{k_1}$ in this 1-ball. Then, look at the set $\{x_n: n \geq k_1\}$. This set still has $x$ as its limit point, wherein you choose $\varepsilon_2=1/2$ and get the corresponding $x_{k_2}$ in the $1/2-$ball of $x$. As such keep going to get a subsequence $x_{k_j}$ that converges to $x$. We know that $\forall \varepsilon>0, \exists n_0$ so that $\forall n,m \geq n_0, d(x_n,x_m)\leq \varepsilon/2$, more specifically, $\forall n,k_j \geq n_0, d(x_n,x_{k_j})<\varepsilon/2$. Moreover, $\forall \varepsilon>0, \exists n'$ so that $\forall k_j \geq n'$ we have $d(x_{k_j},x)<\varepsilon/2$. Adding these two gives us that $x_n \to x$. Therefore, we assume that $\{x_n:n \in \NN\}$ has no limit point. Let $F_m:=\{x_n:n \geq m\}$. Each of these sets, by virtue of having no limit points, are closed. They are, as well, nested and obviously non empty. Moreover, $diam(F_l):=\sup\{d(x_n,x_m): m,n \geq l\}$. If $\varepsilon>0$ is given, there exists a $l_0$ so that $\forall n,m \geq l_0$, we have $d(x_m,x_n)<\varepsilon$ which means $\sup\{d(x_m,x_n):n,m \geq l_0\}\leq \varepsilon$. Hence, $diam(F_n) \to 0$. Hence, from hypothesis, $\cap_{n \in \NN} F_n =\{x\}$. $d(x_n,x)\leq diam(F_n)$, which means that $x_n \to x$. Hence, cauchy complete.  }
\thmp{}{Cauchy completeness $\iff$ completeness}{$\implies)$Let $X$ be a cauchy complete space. That is, every cauchy sequence in $X$ converges in $X$. Let $X \subseteq Y$. If $(z_n)$ is a sequence in $X$, then if it converges, it must converge in $X$. This is because, if it converges outside of $X$, that makes $(z_n)$ a cauchy sequence, which by definition must converge in $X$. Hence, $X$ is closed with respect to $Y$. Since $Y$ was arbitrary, it is closed in every subspace.
\\\\ $\impliedby $) Suppose $X$ is closed with respect to every superspace. Suppose $(x_n)$ is a cauchy sequence in $X$. Consider the completion $X'$ of $X$, that is, that unique superspace of $X$ such that $cls_{X'}(X)=X'$ and $X \hookrightarrow X'$ is an isometric embedding. But since $X$ is closed with respect to $X'$, we have that $X=X'$, and the cauchy sequence in $X$ converges in $X=X'$. Hence $X$ is cauchy complete.  }
\textbf{An alternate version using an intermediate lemma:}
\pf{Let $(x_n)$ be a cauchy sequence in $X$. Let $x \not \in X$. Define $\delta(a,b)=d(a,b)$ if $a,b \in X$. Define $\delta(a,x)=\lim_{n \to \infty } d(a,x_n)$. This limit exists since $d(a,x_n)\leq d(a,x_m)+d(x_n,x_m) \implies d(a,x_n)-d(a,x_m) \leq d(x_m,x_n)$. The far right can be made arbitrarily small for large values of $m,n$, hence $d(a,x_n)$ is cauchy, and hence convergent. Now, look at $\delta(x_k,x)=l_k:=\lim_{n \to \infty} d(x_k,x_n)$. For arbitrarily large vaules of $k$, by virtue of $x_n$ being a cauchy sequence, we can make $l_k$ arbitrarily small. Hence, $\delta(x_k,x) \to 0$ meaning, $x_k \to x$ (under the metric space $X \cup\{x\}, \delta$). All that is left to show is that this function $\delta$ is actually a metric. Let $x \not\in X$ and $y \in X$. We define $\delta(x,y):=\lim_{n \to \infty}d(y,x_n)$. $\delta(x,y'):=\lim_{n \to \infty} d(y',x_n)$. Positivity is obvious since we assert that $x_n$ is actually not convergent in $X$. Let $a,b \in X$. We check triangle inequality for these. Is $\delta(a,b)=d(a,b)\leq \delta(a,x)+\delta(b,x)=\lim_{n\to \infty} d(a,x_n)+\lim_{n \to \infty} d(b,x_n)? $ We have the RHS $\lim_{n \to \infty}( d(a,x_n)+d(b,x_n))$ which would be larger than $d(a,b)$. Now let $a \in X$ and $x=x$. Let $b \in X$. We want to show $\delta(a,x)=\lim_{n\to \infty}d(a,x_n) \leq \delta(a,b)+\delta(b,x)=d(a,b)+\lim_{n \to \infty}(b,x_n)$ In essence we want to show that $\lim_{n \to \infty} d(a,x_n) \leq \lim_{n \to \infty}(d(a,b)+d(b,x_n))$. Notice that the right hand side is always larger than $d(a,x_n)$, which is still preserved in the $n \to \infty $ limit. Hence, triangle inequailty is verified. Hence, $\delta$ is a metric.}
\exm{Two disjoint closed subsets $X,Y$ of $\RR$ that have $d(X,Y)=0$}{Consider the sequence of irrational numbers $x_1,x_2, \cdots$ that is monotonically increasing and divergent (to $+ \infty$). Let this be $X$. Consider another sequence, likewise monotonic increasing and likewise divergent, $y_n$ of rational numbers obeying the property $d(x_k,y_k)<1/k$. We are done.}
\lemp{}{If $A$ is a set in a metric space $X$, and $A$ is closed, then $A$ can be written as the union of the set of isolated points of $A$ and the set of limit points of $A$ (disjoint union).}{That it would be disjoint union is obvious, since isolated points of $A$ are precisely not limit points of $A$. Since $A$ is closed, $A'$ (derived set) is a subset of $A$. $A=(A \setminus A') \cup A'$. $(A \setminus A'):= \{x \in A: \exists \varepsilon_{x} \text{ such that } B_{\varepsilon}(x) \setminus\{x\} \subseteq A^{C} \}$ is precisely the set of isolated points of $A$. We are done.}
\newpage
\section{Compactness}
\defn{Open Conver}{A collection of open sets $G_{\alpha} \subset X$ is an open cover of a set $E$ if $E \subset \cup_{\alpha}G_{\alpha}$}
\defn{Compact Set}{A set $E \subset X$ is said to be \textbf{\emph{Compact}} if Every open cover has a finite subcover. i.e, for every collection of open sets $G_{\alpha}$, if $E \subseteq \cup_{\alpha}G_{\alpha}$, then there would exist a finite sub collection $\{G_{\alpha_1},G_{\alpha_2} \cdots G_{\alpha_k} \}$ of $\{G_{\alpha}\}$ such that $E \subseteq \cup_{i=1}^k G_{\alpha_i}$}
\rmkb{The notion of \emph{Being open} depends largely on the metric space one is talking about. For example, we see that certain sets may be open relative to $Y \subset X$, but not $X$ in itself. This is not the case for compactness though, as shall be seen.}
\thmp{"Compact Relativeness" is conserved. }{Definition: We say $E\subseteq Y \subseteq X$ is compact relative to $Y$ if for every open cover $G_{\alpha}$ open relative to $Y$ we have a finite sub collection $G_{\alpha_k}$ of $G_{\alpha}$ such that $E\subseteq \cup_{j=1}^k G_{\alpha_j}$.\\\\\textbf{Theorem:} $E \subseteq Y \subseteq X$ is compact relative to $Y$ $\iff$ $E$ is compact relative to $X$}{$\implies$)Suppose $E$ is compact relative to $Y$. This means that, for any collection of sets $F_{\alpha}$ which are open relative to $Y$ (i.e, $F_{\alpha}=G_{\alpha} \cap Y$ where $G_{\alpha}$ is an open set in $X$), there exists a finite sub collection $F_{\alpha_1},F_{\alpha_2}\cdots F_{\alpha_k}$ such that $E \subseteq \cup_{i=1}^k F_{\alpha_i}$. Consider an open cover $H_{\alpha}$ of $E$ open relative to $X$. $E\subseteq \cup_{\alpha}H_{\alpha}$, but also, $E\subseteq (\cup_{\alpha}H_{\alpha})\cap(Y)$ since $E$ is subset of $Y$ as well. This implies $E\subseteq \cup_{\alpha}(H_{\alpha}\cap(Y))$. $\{ H_{\alpha} \cap Y \}$ is an open cover of $E$ open relative to $Y$ which means there would be a finite sub collection $\{ H_{\alpha_j}\cap Y: j \in[1,k]\}$ such that $E \subseteq \cup_{j=1}^k (H_{\alpha_j} \cap Y)=(\cup_{j=1}^kH_{\alpha_j})\cap Y$. Since $E$ is a subset of $Y$, we then have $E \subseteq (\cup_{j=1}^kH_{\alpha_j})$ which proves that for an arbitrary open cover open relative to $X$, we have a finite subcover.
\\\\ $\impliedby$) Suppose $E$ is open relative to $X$. Consider an open cover of $E$ open relative to $Y$, which is $\{F_{\alpha}\}$. This means that $F_{\alpha}=G_{\alpha} \cap Y$ for $G_{\alpha}$ open relative to $X$. $E\subseteq \cup_{\alpha} F_{\alpha}=(\cup_{\alpha} G_{\alpha})\cap Y$. Since $E$ is a subset of $Y$, we have $E \subseteq (\cup_{\alpha} G_{\alpha})$. Therefore, there would be a finite subcollection of $\{ G_{\alpha} \}$, $\{G_{\alpha_1},G_{\alpha_2}\cdots G_{\alpha_k} \}$ such that $E \subseteq \cup_{i=1}^k G_{\alpha_i}$. This means, $E\subseteq \cup_{i=1}^k G_{\alpha_i} \cap Y=\cup_{i=1}^k F_{\alpha_i}$. Hence, for every open cover open relative to $Y$, there exists a finite subcover.

}
\newpage
\rmkb{Note that the above theorem "compact-relativeness is conserved" did not use any property of metric spaces, like distances or balls, so this theorem holds in any general Topological space.}

\fact{Every finite set in $X$ is compact}\pf{Consider an open cover $G_{\alpha}$ for finite set $E$. This means that, for every point $x_1,x_2,\cdots,x_k$ in $E$, there would exist some $\{\alpha_1,\alpha_2, \cdots \}$ collection of "$\alpha$-s" that is utmost finite, such that $x_j \in G_{\alpha_j}$. Simply take the union of $G_{\alpha_j}$ to get a finite subcover.
}
\thm{Alternate definition for compactness}{A set $E$ is compact if for every closed collection of sets $K_{\alpha}$ such that $\cap_{\alpha} K_{\alpha} \subset E^c$, we have a finite subcollection $\{K_{\alpha_1},K_{\alpha_2} \cdots K_{\alpha_p}\}$ such that $\cap_{i=1}^p K_{\alpha_i} \subset E^c$}

\thmp{}{Closed balls in $X$ are closed}{Consider $B_{[\varepsilon]}(p):=\{x \in X: d(x,p) \leq \varepsilon\}$. $B^c=C:=\{x \in X, d(x,p)=t_{xp}>\varepsilon\}$. Consider an arbitrary point $x \in C$. We have $d(x,p)=t_{xp}>\varepsilon$. Find, from density, a $\delta$ such that $t_{xp}>\delta>\varepsilon$. Let $d(x,y)<\delta-\varepsilon $. We then have from triangle, $d(y,p)\geq d(x,p)-d(x,y)>t_{xp}-(\delta-\varepsilon)>\varepsilon$. Hence, $y$ is also in $C$. Therefore, $C$ is open, which means $B$ is closed.}
\begin{figure}[h]
    \centering
    \includegraphics[width=0.5\linewidth]{Images/ballz.png}
    \caption{Figure for the proof: Closed balls are closed}
    \label{Closed balls are closed}
\end{figure}
\defn{$diam(A)$}{$$diam(A):=\sup\{d(x,y):x,y \in A\}$$ It is finite if and only if it is bounded. It would be infinite if unbounded. }
\lemp{}{If $x_n$ is a cauchy sequence and $\{x_n\}$ the set contains a limit point, then the cauchy sequence converges.}{$\forall \varepsilon/2>0$, $\exists n_0$ so that $\forall n,m \geq n_0$, $d(x_n,x_m)<\varepsilon/2$. Suppose it has a limit point $x$. Around every $\varepsilon/2$ ball of $x$, there exists infinite points of $\{x_n\}$. Choose $x_1$, and then make $\delta<\min\{d(x_1,x)\}$ and get a corresponding $x_{n_2}$. Then choose $\delta<\min\{x_{n_2},x_1\}$ and keep going to create a subsequence of $\{x_n\}$ that converges to $x$. 
$\forall \varepsilon/2>0$ $\exists n'$ so that $\forall n_k \geq n' $ we have $d(x_{n_k},x)<\varepsilon/2$. Therefore, $\forall \varepsilon>0$, $\exists q=\max{n',n_0}$ so that $\forall n,n_r \geq q$, we have $d(x_{n},x_{n_s})<\varepsilon/2$ and $d(x_{n_s},x)<\varepsilon/2$, Adding these two we get the desired result.  }
\thmp{}{A metric space $X$ is complete if and only if for every sequence of nested non-empty closed sets $\{F_n:n \in \NN\}$ such that $\lim_{n \to \infty}$diam($F_n$)=$0$, there exists one point $x \in \cap_{i=1}^{\infty} F_i$}{$\implies$) Say $X$ is complete. Every cauchy sequence converges. Consider an arbitrary nested sequence of closed sets so that $F_n \subseteq F_{n-1}$. Choose $x_{n_1}=x_1 \in F_1$. Choose $x_2 \in F_2\setminus (F_1)$ (if it exists). If not, move on to the next set, and so on. If every set contains only one point, then we are done. If not, we are able to find $x_{n_2} \in F_{n_2}$ such that $x_{n_2} \neq x_{n_1}$. As such create a sequence $x_{n_k} \in F_{n_k}$ where $n_{k}<n_{k+1}$ (which gives $F_{n_{k+1}}\subset F_{n_k}$). Is $x_{n_k}$ a cauchy sequence? Let $\varepsilon>0$. Since $diam(F_{n_k})\to 0$, we have that there exists some $k \in \NN$ such that for every $n_k \geq k$, we have $0 \leq diam(F_{n_k})<\varepsilon$. Let $q,p \geq $ this $k$. That means $x_{n_q}$ and $x_{n_p}$ are both in $F_{n_k}$ which means $d(x_{n_q},x_{n_p})\leq diam(F_{n_k})<\varepsilon$ which means this sequence is cauchy, hence convergent. Hencce a limit $x$ exists. Is $x \in \cap_{i=1}^{\infty}F_i$? If so, we are done. If not, it must exist in $\cup_{i=1}^{\infty}F^C_i$. This is a countable union of open sets $F^c_i$ which means it is finally in an open set. Outside any $F_j$, only finite points of the sequence exists. This immediately contradicts the previous result, since $x$ is a limit point of the sequence, but it is in some $F_j^c$ which contains only finite points of the sequence so it is impossible for a particular (chosen) $\varepsilon-$ball to contain infinite points of $\{x_{n_k}\}$. Hence, we are done. $x \in \cap_{i=1}^{\infty} F_i$. Moreover, since $diam(F_n) \to 0$, it is trivial to see that only one point survives the intersection. (If there are two points,we can always make epsilon smaller than their fixed distance). 
\\\\ $\impliedby)$ Suppose $\{x_n\}$ is a cauchy sequence, i.e, $\forall \varepsilon>0$ $\exists n_0$ so that for all $m,n \geq n_0$, $d(x_n,x_m)<\varepsilon$. Look at $S_m:=\{x_n:n \geq m\}$. If any of these sets has a limit point, then we would be done from the previous lemma. So we assume these all dont have a limit point. Hence they are closed, nested sets. Moreover, $\forall \varepsilon>0$, $\exists n_0$ so that $\forall n,m \geq n_0$ (or rather, for any two points $x$ and $y$ in $S_{n_0}$) we have $d(x_n,x_m)<\varepsilon$(or $d(x,y)<\varepsilon$) which means $diam(S_{n_0})<\varepsilon$. This means $diam(S_n) \to 0$. This means that $\cap_{i=1}^{\infty}S_i=\{x\}$ for some element $x$. $x$ is in every $S_i$. Let $\varepsilon>0$. There exists $S_{n_{\varepsilon}}$ so that $diam(S_{n_{\varepsilon}})<\varepsilon$ which means $d(x_{n_{\varepsilon}},x)<\varepsilon $ which means $x_{n_{\varepsilon}}$ forms a convergent subsequence, which means the sequence is finally convergent from previous lemma.  }
\thmp{}{Compact sets are closed.}{\textbf{Method 1:}\\\\Let $E$ be compact. Consider a point $p \in E^c$. Let $\varepsilon_x$ be the "half" distance between a point $x \in E$ and $p$. Therefore, $B_{\varepsilon_x}(x)$ is completely outside $B_{\varepsilon_x}(p)$. Consider $\cup_{x \in E} B_{\varepsilon}(x) $ which is an open cover for $E$. This means there is a finite subcover 

$\{B_{\varepsilon_{x_1}}(x_1), B_{\varepsilon_{x_2}}(x_2), B_{\varepsilon_{x_3}}(x_3) \cdots, B_{\varepsilon_{x_k}}(x_k) \}$ such that $E \subset \cup_{i=1}^k B_{\varepsilon_{x_i}}(x_i)$. $B_{\varepsilon_{x_i}}(p)$ does not intersect with $B_{\varepsilon_{x_i}}(x_i)$. Therefore, $\cap_{i=1}^kB_{\varepsilon_{x_i}}(p)$ does not intersect with any $B_{\varepsilon_{x_i}}(x_i)$ for any $i$. Hence, it does not intersect with $\cup_{i=1}^k B_{\varepsilon_{x_i}}(x_i)$ which means $\cap_{i=1}^kB_{\varepsilon_{x_i}}(p)$ lies completely outside $E$. If we choose $\delta< min\{\varepsilon_{x_1},\varepsilon_{x_2} \cdots \varepsilon_{x_k}\}$, we would have $B_{\delta}(p) \subseteq \cap_{i=1}^kB_{\varepsilon_{x_i}}(p)$. This means that, for $p$ outside $E$, there would exist a $\delta$ such that the $\delta$-ball around $p$ is fully contained in $E^c$. This means that $E^c$ is open, hence, $E$ is closed.
\\\\ \textbf{Method 2:}\\\\ Consider $E$ to be compact, i.e, for every closed collection $\{F_{\alpha}\}$ such that $\cap_{\alpha} F_{\alpha} \subset E^c $, there exists a finite sub collection $\{F_{\alpha_1},F_{\alpha_2} \cdots F_{\alpha_k} \}$ such that $\cap_{j=1}^k F_{\alpha_j} \subset E^c$. Consider a point $p$ outside $E$, i.e, in $E^c$. Notice that $\cap_{\varepsilon \in \RR^+} B_{[\varepsilon]}(p)=\{ p\}$ which is in $E^c$. This would be a collection of closed sets whose intersection falls completely inside $E^c$. Hence, there would exist a finite subcollection such that $\cap_{j=1}^k B_{[\varepsilon_j]}(p)\subset E^c$ which means there would exist a neighbourhood around $p$ which is completely in $E^c$. Hence, $E^c$ is open, and $E$ is closed. 

}
\fact{$\phi$ and $X$ are both open and closed.}
\thmp{}{Closed subsets of compact sets are compact}{Consider $K \subset E$ where $E$ is compact and $K$ is closed. $K^C$ is, therefore, open. Consider an arbitrary open cover $\{F_{\alpha}\}$ for $K$. since $K\subseteq \cup_{\alpha} F_{\alpha}$, and $K^c$ is open, we have $X=\cup_{\alpha} F_{\alpha} \cup K^c$ which means $E \subset \cup_{\alpha}F_{\alpha} \cup K^c$. Since $E$ is compact, there would exist a finite subcover such that $E \subset \cup_{j=1}^n F_{\alpha_j} \cup K^c$. We then have $K \subset \cup_{j=1}^n F_{\alpha_j} \cup K^c$, which would mean $K \subset \cup_{j=1}^n F_{\alpha_j}$, whence, we see that $K$ is compact.}
\cor{If $F$ is closed, and $K$ is compact, then $F \cap K$ is compact}
\fact{A compact set is bounded}\pf{Consider (WLOG, a non empty compact set $E$) and an arbitrary point $q$ in $X$. $B_{\varepsilon}(x)$ for every $\varepsilon>0$ forms an open cover for $E$ (since it is basically $X$). Which means there is a finite subcover, i.e, a number $\varepsilon_0>0$ such that $E \subset B_{\varepsilon_0}(p)$ which makes $E$ bounded.}

\thmp{}{Finite union of compact sets is compact}{Let $K_1,K_2,\cdots K_r$ be $r$ compact sets. Let $K=\cup_{i=1}^rK_i$. Consider an open cover $F_{\alpha}$ whose union subsumes $K$. We have that, for every $i \leq r$, $K_i \subset \cup_{\alpha}F_{\alpha}$. Since each $K_i$ is compact, there exists a finite of $F_{\alpha}$ whose union subsumes $K_i$. For each $i=1,$ to $r$, we have a finite subcollection, therefore, taking the union of all these finite subcollections gives us a finite subcollection which subsumes whole of $K$. Hence $K$ is compact.

}
\rmkb{Note that finiteness in the above theorem is important. This is because, each compact may have finite subcollection, but at the end, the union of all these finite collections will be countable, not finite.}

\thmp{}{If $\{K_{\alpha} \}$ is a collection of compact sets such that for every finite subcollection $\{K_{\alpha_j}: 1\leq j\leq k\}$ we have that $\cap_{j=1}^k K_{\alpha_j} \neq \phi$. Then $\cap_{\alpha} K_{\alpha} \neq \phi$. In pithy words:\\\\\centering{{"If you have a collection of compact sets for which every finite subcollection's intersection is non-empty, the intersection of the whole collection is non empty"}{\\-\textit{Krishna, to Arjuna}}} }{Suppose, on the contrary, let $\cap_{\alpha} K_{\alpha}=\phi$ which means $\cup_{\alpha} K^c_{\alpha}=X$ which means for every for some $\alpha_0$, we have $K_{\alpha_0} \subset \cup_{\alpha} K^c_{\alpha}$, where $\cup_{\alpha} K^c_{\alpha}$ is an open cover of $K_{\alpha_0}$. This implies that there exists a finite subcollection $\{K^C_{\alpha_1},K^C_{\alpha_2} \cdots K^C_{\alpha_r}\}$ such that $K_{\alpha_0}\subset \cup_{j=1}^r K^C_{\alpha_j} \implies \cap_{j=1}^r K_{\alpha_j} \subset K^C_{\alpha_0}$. But this means $\cap_{j=1}^r K_{\alpha_j} \cap K{\alpha_0}=\phi$, which is absurd since all finite intersection is non empty.

}
\cor{If $K_1,K_2,\cdots$ is a sequence of non-empty compact sets such that $\cdots K_n \subset K_{n-1} \cdots K_3 \subset K_2 \subset K_1$, then $\cap_{i=1}^{\infty} K_i$ is non empty.

}
\thmp{Compactness $\implies$ Limit point Compact}{If $K$ is a compact set and $E$ is an infinite subset of $K$, then $E$ has a limit point in $K$}{Suppose that $E$ has no limit point in $K$. Since $K$ is closed, $E$ must have no limit points. Hence, $E$ is closed. Since closed subsets of compact sets are compact, $E$ is compact. If no point of $E$ is a limit point of $E$, then $\forall x \in E$, $\exists \varepsilon_x>0$ such that no point of $E$ apart form $x$ itself falls into the $\varepsilon_x$-ball of $x$. Consider the open cover $\{B_{\varepsilon_x}(x): x\in E\}$ of $E$. This has a finite subcover $\{B_{\varepsilon_{x_1}}(x_1), B_{\varepsilon_{x_2}}(x_2), \cdots B_{\varepsilon_{x_l}}(x_l)\}$. We see that $E \subseteq \cup_{j=1}^l B_{\varepsilon_{x_j}}(x_j)$. But since for every $\varepsilon_{x_j}$-ball around $x_j$, no point in $E$ except $x_j$ resides, $\cup_{j=1}^lB_{\varepsilon_{x_j}}(x_j)$ will have utmost finite points. Since an infinite set $E$ cannot be the subset of a finite set, we have a contradiction. }

\defn{k-cell}{a $k$-cell, $E$ is a set in $\RR^k$ such that $E:=\{\vec{x}=(x_1,x_2,\cdots,x_k) \in \RR^n: a_j\leq x_j \leq b_j \text{ for given }a_j \text{ and } b_j \text{ for every } 1\leq j\leq k  \}$\\\\ A $k$-cell is basically a $k$ dimensional cuboid.  }

\thmp{}{$k$-cells are closed}{Consider a $k$-cell $E$. Consider a point $z$ not in $E$, i.e, $\exists j_0$ such that either $z_j<a_j$ or $z_j>b_j$. WLOG, take the case of $z_j<a_j$. Let $0<\delta<(a_j-z_j)$. Consider a point $q$ in the $\delta$-ball around $z$. i.e, $d(z,q)<\delta \implies \sqrt{(z_1-q_1)^2+(z_2-q_2)^2+\cdots+(z_k-q_k)^2}<\delta \implies (z_1-q_1)^2+(z_2-q_2)^2+\cdots+(z_k-q_k)^2 <\delta^2<(a_j-z_j)^2 \implies 0<(q_j-z_j)^2<(a_j-z_j)^2$ $\implies q_j<a_j$. Hence $q \not\in E$, which implies there exists, for every $x$ in $E^c$, a $\delta$ for which the $\delta$-ball around $x$ is fully contained in $E^c$ which means $E^c$ is open. This implies $E$ is closed. Same argument applies for the case where $z_j>b_j$.}

\thmp{}{Closed intervals in $\RR$ are compact}{Let $\II$ be, WLOG, $[-a,a]$. Suppose it is not compact. i.e, There is an open cover $G_{\alpha}$ of $\II$ such that there exists no finite subcover. $\forall x \in \II$, $\exists \alpha_x$ such that $x \in G_{\alpha_x}$ and $\exists \varepsilon_x$ such that $B_{\varepsilon_x}(x) \subset G_{\alpha_x}$. $\cup_{x} B_{\varepsilon_x}(x)\subset \cup_{\alpha} G_{\alpha}$ is an open cover for $\II$. note that, if no finite subcover for $G_{\alpha}$ exists, then no finite subcover for $B_{\varepsilon_x}(x)$ exists either. So we can safely work with $B_{\varepsilon_x}(x)$. Split the interval into two halves, $[-a,0]$ and $[0,a]$. One of these intervals is not finitely covered by $B_{\varepsilon_x}(x)$, for if not, the whole thing would be finitely covered. let that interval which is not finitely covered be $\II_1$. This interval's size is $a$. Split this interval into two again. Yet again, one of the halves must not be finitely covered, for if not, $\II_1$ would be finitely covered, which is contradictory. Let this interval be $\II_2$. This is of size $\frac{a}{2}$. Yet again, keep doing this process to obtain a sequence of intervals $\II_j$, sized $\frac{a}{j}$, which are not finitely covered. These are nested intervals, non empty, and closed. From nested intervals theorem, we see that a point $\xi$ exists in $\cap_{j=1}^{\infty}\II_j$. $\xi$ is a point in $\II$, and there is a corresponding $\varepsilon_{\xi}$. Consider that $j_0$ for which $\frac{a}{j_0}<\frac{\varepsilon_{\xi}}{2}$. We know from Archimedean such a $j_0$ exists. This means that the interval $\II_{j_0}$ containing $\xi$, sized $\frac{a}{j_0}$, is completely inside the $\varepsilon_{\xi}$-ball around $\xi$, which means it is finitely covered. Contradiction. Hence, $\II$ is compact.}
\cor{Since intervals of the form $[-a,a]$ are compact, every closed interval of the form $[a,b]$ is compact since it would be a closed subset of an interval of the form $[-x,x]$.}
Generalisation:
\thmp{}{$n$-cells are compact}{Consider $K:=\{ \vec x \in \RR^n: -a \leq x_j \leq a; \forall j \leq n\}$ to be non-compact. There is an open cover $G_{\alpha}$ of $K$ such that there exists no finite subcover. $\forall x \in K$, $\exists \alpha_x$ such that $x \in G_{\alpha_x}$ and $\exists \varepsilon_x$ such that $B_{\varepsilon_x}(x) \subset G_{\alpha_x}$. $\cup_{x} B_{\varepsilon_x}(x)\subset \cup_{\alpha} G_{\alpha}$ is an open cover for $K$. note that, if no finite subcover for $G_{\alpha}$ exists, then no finite subcover for $B_{\varepsilon_x}(x)$ exists either. So we can safely work with $B_{\varepsilon_x}(x)$. Till here, everything is the same as the $1$-d case. Note that here, the $n$-cell is constructed by taking the cartesian product of $n$- intervals in $\RR$ of the kind $[-a,a]$. Construct $2^n$ subdivisions of $K$ by halving each interval $[-a,a]$ in the construction of $K$. The total number of subdivisions we make would be $2 \times 2 \times \cdots \times 2$, n times (simple combinatorial argument: for each $i$, there exists $2$ choices, the two half intervals, for crossing. From $i=1$, you have 2 choices, likewise, $j=2,3,\cdots n$). We assume that atleast one of these $2^n$ subdivisions are not finitely covered by $\{ B_{\varepsilon_x}(x) \}$. We let this one be $K_1$, whose each interval size is now $a$. Subdivide this yet again into $2^n$ subsets, and assert that one of these subdivisions is not finitely covered. Call this $K_2$, whose each interval is of size $\frac{a}{2}$. Construct a sequence of sets $K_j$, each of whose intervals are sized $\frac{a}{j}$. Each $K_j$ is closed and non empty, hence compact, and are nested. Therefore, $\cap_{j=1}^{\infty} K_j \neq \phi$. Let $p \in \cap_{j=1}^{\infty} K_j$. For this $p$, there would exist a $\varepsilon_p$ and the corresponding ball $B_{\varepsilon_p}(p)$. We require one of our $K_j$ $n$-cell to fall into this $\varepsilon_p$ ball. Let $\delta$ be smaller than $\frac{\varepsilon_p}{2}$. Let $p$ be the centre of the $\delta-$ball. Let $p$ be the centre of the $n$-cube $H$ in the following construction: Consider $w$ to be the side length of $H$. We require the diagonal length $w\sqrt{n}=\delta$, which gives us $w=\frac{\delta}{\sqrt{n}}$. Consider $H$ such that each side is the interval $[p_j-\frac{w}{2},p_j+\frac{w}{2}]$. This would force $p$ to fall in the centre of the $n$-cube $H$. This cube is fully contained in the $\delta$ ball of $p$ which is contained in the $\varepsilon_p$ ball of $p$. We consider that $n$ cell $K_j$ for which each side $\frac{a}{j}<w$. This can be found, and hence, the this $K_j$ cell is finitely covered by $B_{\varepsilon_p}(p)$, which is absurd. Hence, $K$ is compact.}

\begin{figure}[h]
    \centering
    \includegraphics[width=0.5\linewidth]{Images/kcellz.png}
    \caption{Figure for the proof: $n$-cells are compact.(The $\varepsilon_p$-ball around $p$, and the $n$-cell construction)}
    \label{fig:ncellz}
\end{figure}

\rmkb{We proved the result for $k$-cells of the kind $[-a,a]^k$, but it is easily generalised by noting that arbitrary $k$-cells are contained in some $k$-cell of the above kind. By virtue of being closed, they also are compact.}
\thmp{Heine-Borel}{Given a set $E \subset \RR^n$, the following are equivalent:
\begin{enumerate}
    \item $E$ is closed and bounded
    \item $E$ is compact
\end{enumerate}}{$\impliedby$) We know that all compact sets are closed.\\\\
$\implies$) If $E \subset \RR^n$ is closed and bounded, it is contained in some $n$-cell, which is compact. By virtue of being a closed subset of a compact set, $E$ is also compact.}
\thmp{}{If $\{ x_n\}$ is a sequence in $X$ convergent to $x\in X$, then the set $\{x_n\}$ has only one limit point, which is $x$.}{That $x$ is a limit point is clear. We note that $\forall \varepsilon>0$, $\exists n_0$ such that $\forall n \in \NN, n\geq n_0$ we have $d(x_n,x)<\varepsilon$. i.e, beyond a particular $n_0$, every point of $\{X_n\}$ falls in the $\varepsilon$-ball of $x$. Therefore, only finite points lie outside this $\varepsilon$-ball of $x$. Suppose it has another limit point $y$, other than $x$. Therefore, there would exist a $\delta$ such that the $\delta$-ball around $y$ lies completely outside the $\delta$-ball around $x$. This means that, only finite points of $x_n$ lie in the $\delta$ ball of $y$, making it unviable to be a limit point.}

Extending Heine Borel we have:
\thmp{(Extension)}{For a subset $E \subset \RR^n$, the following are equivalent: 
\begin{enumerate}
    \item $E$ is closed and bounded
    \item $E$ is compact
    \item every infinite subset $K$ of $E$ has a limit point in $E$
\end{enumerate}}{$(1)\implies(2)$) Heine Borel\\\\
$(2)\implies(3)$) Already seen\\\\
$(3)\implies(1)$) Let us assume that $E$ is either not closed, or not bounded. We start by assuming it is not closed. Which means that $\exists q$ outside $E$ such that there exists a sequence in $E$ that converges to $q$. We take this sequence $\{x_n\}$ as our infinite set, and we see that, from the previous theorem, this has only one limit point $q$, which lies outside $E$. Hence, there exists an infinite set $\{x_n\}$ which has no limit point in $E$. 
\\\\ Suppose that $E$ is unbounded. We then have that, $\forall x \in X$, $\forall \varepsilon>0$, $\exists y \in E$ such that $d(x,y)> \varepsilon$. Fix some $x_0$ in $X$. Choose some $y_0$ that is a distance $z_{00}=d(y_0,x_0)$ away from $x_0$. Look if there is a point $y_1$ so that its distance from $x_0$ is more than $z_{00}$ but less than $2(z_{00})$. If it doesn't exist, check for less than $3(z_{00})$. Find some $k_1(z_{00})$ so that distance of $y_1$ to $x_0$ is more than $z_{00}$ but less than $k_1(z_{00})$. Same way, for $y_2$, find $y_2$ so that its distance from $x_0$ is more than $k_1{(z_{00})}$ but less than some other $k_2(z_{00})$. Inductively, find $y_j$ whose distance is more than $k_{j-1}(z_{00})$ but less than $k_j(z_{00})$. Note that $1<k_1<k_2<\cdots$. Hence, for any $\varepsilon$-ball around $x_0$, only finite $y_j$ exists in that ball, since there would exist some $k_{q}(z_{00})$ and $k_{q-1}(z_{00})$ between which $\varepsilon$ lies. And inside $k_{q-1}(z_{00})$ ball around $x_0$, utmost finite points $y_j$ exists. Hence, $x_0$ is clearly not a limit point for the set of $y_j$-s. Consider any other point $a \in X$. For some every $\varepsilon$-ball around $x_0$, only finite points exists. For some, perhaps larger $\delta$-ball around $a$, a chosen $\varepsilon$-ball around $x_0$ gets subsumed into the $\delta$-ball around $a$. This implies that only finite points of $y_j$-s exists in the $\delta$-ball around $a$ as well, making $a$ a non viable limit point. We see that, for this infinite subset $\{y_j\}$ of $E$, no limit point exists.  }
\begin{figure}[h]
    \centering
    \includegraphics[width=0.5\linewidth]{Images/limptcompact.png}
    \caption{Figure for proof: Lim point compact $\implies$ closed+bounded. (The construction of an unbounded sequence)}
    \label{fig:lptcpt}
\end{figure}
\rmkb{In the previous proof, we note that (3), which is called Limit Point Compactness, implies (1) Closed and Bounded, in any metric space, not just $\RR^n$, as we see in the proof, no property of $\RR^n$ was used.\\\\
\textbf{Spoiler Alert:} In any metric space, Limit Point Compact $\iff$ Compact }

\thmp{Weierstrass Theorem}{Every Bounded, infinite set in $\RR^n$ has a limit point in $\RR^n$.}{If a set is bounded in $\RR^n$, it is the subset of a compact set (i.e, a closed and bounded set). From the previous equivalence, an infinite subset $E$ of a compact set has a limit point in the compact set, which means the bounded, infinite set we have has a limit point in the compact set that contains it, hence, it has a limit point in $\RR^n$.}
\rmkb{The above "Weierstrass Theorem" is just the "Bolzano-Weierstrass" Theorem we saw in sequences. Actually, the "Bolzano-Weierstrass" Theorem is a direct corollary of the more general "Weierstrass Theorem". Let $\{x_n\}$ be any sequence in $\RR$ that is bounded. This means that this sequence is the subset of a compact set, hence, has a limit point in $\RR$. This implies, a subsequence of $\{x_n\}$ converges in $\RR$. Hence, every bounded sequence has a convergent subsequence. }

\fact{Let $X=\RR^n$. The closure of any open ball is the corresponding closed ball.}
\pf{Consider $B:=B_{\delta}(x_0):=\{y \in \RR^n: ||y-x_0||<\delta \}$. Let $z_0$ be a point on the rim of $B$, i.e $d(x_0,z_0)=\delta$. Such a point obviously exists. Consider $\vec\gamma(t)=t\vec z_0 +(1-t)\vec x_0$ with $t \in(0,1)$.
For every $t \in(0,1)$, $\vec \gamma(t)$ belongs in $B$. To see this, consider $||\vec\gamma(t)-\vec x_0||=||t\vec z_0+(1-t)\vec x_0-\vec x_0||=||t(\vec z_0- \vec x_0) ||=t||\vec z_0-\vec x_0||<t(\delta)<\delta$. Suppose we are given an arbitrary $\eta>0$. Does there exist a $t \in(0,1)$ so that $\vec \gamma(t)$ belongs in the $\eta$-ball of $\vec z_0$? i.e, we need a $t$ so that $||\vec \gamma(t)-\vec z_0||=||t\vec z_0+(1-t)\vec x_0 -\vec z_0||=||(1-t)(\vec z_0-\vec x_0)||<\eta$ $\implies (1-t)||\vec z_0 -\vec x_0||<\eta \implies 1-t< \frac{\eta}{\delta} \implies 1-\frac{\eta}{\delta}<t<1$. Such a $t$ exists for every $\eta$. Hence, $\vec z_0$ is a limit point of $B$ (by virtue of there existing a sequence of $\vec \gamma(t_j)$ that converges to $\vec z_0$).
Hence, every point on the rim is a limit point. Moreover, no point $w$ so that $d(w,x_0)>\delta$ is a limit point of $B$, since there would exist an $\varepsilon$-ball around $w$ so that no point of $B$ falls into it (from openness). Hence, closure of $B$ is the corresponding closed ball, in $\RR^n$.

}
\defn{Dense sets}{A set $A \subseteq X$ is said to be dense in $X$ if every point of $x$ is a limit point of $A$.}
\thmp{}{$A$ is dense in $X$ if and only if $\overline{A}=X$ if and only if for every open set $U$ of $X$, $A \cap U \neq \emptyset$}{$\implies$) Let $A$ be dense in $X$. Let $U$ be an open set such that $A \cap U=\emptyset$ which means $A$ is completely contained in $U^C$ which is a closed proper subset of $X$, which is a contradiction since $\overline{A} \subseteq U^C$ since it is the smallest closed set that contains $A$. Contradiction.
\\\\$\impliedby$)Suppose $A $ is not dense in $X$, this means $\overline{A}\subset X$ which means $(\overline{A})^C$ is a non empty open set in $X$ that necessarily does not intersect with $A$. Contradiction.}

\section{Perfect Sets}
\defn{Perfect Set}{A set $E \subset X$ is perfect if every point of $E$ is a limit point of $E$, and $E$ is closed}
\thmp{}{Perfect subsets in $\RR^n$ are uncountable.}{Suppose $E$ is a perfect set in $\RR^n$ but is countable. i.e, it can be enumerated as $E=\{x_1,x_2,\cdots \}$.
\\\\ Choose $x_1$, and $\varepsilon_0=1$. Let $V_0$ denote the $\varepsilon_0$-ball around $x_1$. This ball is non empty, moreover, $\bar{V_0} \cap E$(which is the corresponding closed ball of $V_0$) is non empty, and is compact by virtue of being closed and bounded. Inside, $V_0 \cap E$, there exists infinite points of $E$, since $x_1$ is a limit point of $E$.
\\\\ Choose an arbitrary point $z_1$ in $V_0$ that is not $x_1$. Now let $\varepsilon_1<d(x_1,z_1)$. Let $V_1$ be the $\varepsilon_1$-ball around $z_1$. Notice the following: $z_1$ is a limit point of $E$, hence, there are infinite points of $E$ in $V_1$. $x_1$ is not in $\bar{V_1}$. $\bar{V_1}\cap E$ is closed, bounded and non empty, hence Compact. 
\\\\ Choose a point $z_2$ in $V_1$ that is not $x_2$, and let $\varepsilon_2<min\{\varepsilon_1, d(x_2,z_2)\}$. Let $V_2$ be the $\varepsilon_2$-ball around $z_2$. Note that, $x_2$ is not in $\bar{V_2}$. Also note yet again that there are infinitely many points of $E$ in $V_2$. It is crucial to note now that $\bar{V_2}\cap E \subset \bar{V_1}\cap E\subset \bar{V_0} \cap E$.
\\\\Suppose you have already constructed $V_k$ by finding $z_k$ in $V_{k-1}$ that is not $x_k$ and an $\varepsilon_k<min{d(z_k,x_k),\varepsilon_{k-1}}$ such that $x_k \not\in \bar{V_k}$, $\bar{V_k}\cap E$ is compact, non empty and $\bar{V_k}\cap E \subset \bar{V_{k-1}}\cap E \cdots$.
\\\\Now, choose $z_{k+1} \neq x_{k+1}$, inside $V_k$. Choose $\varepsilon_{k+1}<min\{d(z_{k+1},x_{k+1}),\varepsilon_k\}$. Let $V_{k+1}$ be the $\varepsilon_{k+1}$-ball around $z_{k+1}$. Yet again, we see that $\bar{V_{k+1}} \cap E$ is non empty, $x_{k+1}$ is not in $\bar{V_{k+1}}$, and $\bar{V_{k+1}}\cap E \subset \bar{V_{k}}\cap E$. Hence, we have a sequence of non empty, nested compact sets. This implies that $\exists \xi \in E \subset \RR^n$ such that $\xi \in \cap_{i=1}^{\infty} (\bar{V_i}\cap E)$. Is $\xi$ any one of $x_j$ enumerated? No, because if it was, from the construction, $x_j$ would not belong in $V_j$. Hence, $\xi$ is not in the enumeration of $E$. Contradiction.}
\begin{figure}[h]
    \centering
    \includegraphics[width=1.0\linewidth]{Images/perfectsetz.png}
    \caption{Figure: Perfect sets are uncountable. Construction of the nested sequence of compact sets by choosing $z_k\neq x_k \in V_{k-1}$.}
    \label{fig:perfectunctbl}
\end{figure}
\newpage
\rmkb{It is easily seen that, closed intervals in $\RR$ are perfect: From density theorem, for every point in $I$, there would exist a sequence of rationals converging to that point. Moreover, closed intervals in $\RR$ are closed since closed balls in metric spaces are closed. Therefore, we see that intervals are uncountable.}
\subsection{The Cantor Set}
The following is the construction of an uncountable, perfect set that contains no intervals: The Cantor Set.
\\\\ Let $I_0=[0,1]$. size of the interval(s) in $I_0$ is $1$, and there are $2^0=1$ intervals.
\\\\ Let $I_1=[0,\frac{1}{3}]\cup[\frac{2}{3},1]$ be constructed by trisecting $I_0$ and tossing the middle one. Here, we have each interval sized $\frac{1}{3^1}$, and there are $2^1=2$ intervals total.
\\\\Let $I_2=[0,\frac{1}{9}]\cup[\frac{2}{9},\frac{3}{9}]\cup[\frac{6}{9},\frac{7}{9}]\cup[\frac{8}{9},1]$ be generated by taking each of the two sub intervals in $I_1$, trisecting them, and tossing the middle one, and joining them finally. We have each interval sized $\frac{1}{3^2}$ and there are $2^2=4$ intervals total.
\\\\Inductively keep making these trisections+tossings to make a sequence of closed, nested intervals (Compact, too) $I_k$, each containing $2^k$ intervals each of size $\frac{1}{3^k}$.
\\\\ Finally, define the Cantor set $P$ as $$P:= \big.\cap_{i=1}^{\infty}I_i $$ Note that $P$ is compact since it is the closed subset of a compact set. It is also non empty by virtue of being the intersection of a sequence of nested, non empty, compact sets.
\\\\Note that, no interval of the kind $[a,b]$ exists in the Cantor Set. The size of each interval in $I_j$ is $\frac{1}{3^j}$. We can find $j$ so that $\frac{1}{3^j}<b-a \implies \frac{1}{b-a}<3^j \implies \log_{3}(\frac{1}{b-a})<j$. For such $I_j$, we notice that $[a,b]$ has "inbetween" points that doesn't exist in any of $I_j$'s intervals. Hence, taking the intersection, these "inbetween" terms don't survive. Hence, no intervals exist.
\thmp{}{The Cantor set $P$ is perfect.}{We already know that the Cantor set is closed. We need to show that every point in the cantor set is a limit point. First, observe that, for any $I_k$, if $z$ is the end point of any of the sub interval of $I_k$, it survives the $\infty$-intersection. This is because, after $I_k$-s trisection, the end points still stay endpoints. Let $\xi$ be any point in the cantor set, which means it is a point in every $I_k$. Let $\delta>0$ be given. Consider the interval $(\xi-\delta,\xi+\delta)$. This interval is sized $2\delta$. $\xi$ exists in one of the sub intervals of $I_k$ for all $K \geq k_0$ for some $k_0$. Choose $j$ so that $\frac{1}{3^j}<\delta$. Then, the interval in $I_j$ containing $\xi$ would fall completely inside $(\xi-\delta,\xi+\delta)$. Choose $q$ as one of the end points of this sub interval of $I_j$. Therefore, $\forall \xi \in P$, $\forall \delta>0$, $\exists q \in P, q \neq \xi$ so that $q\in (\xi-\delta,\xi+\delta)$. Therefore, every $\xi \in P$ is a limit point of $P$. Hence, $P$ is perfect.}

\section{Connected Sets}
\defn{Separated Sets}{$A \subset X$ and $B \subset X$ are said to be \emph{separated} if $\bar{A} \cap B=
\bar{B}\cap A=\phi$, i.e, they are disjoint and no point of one, is the limit point of the other.}
\defn{Connected Set}{A set $E\subset X$ is said to be \emph{connected} if it is \emph{not} the union of two non-empty separated sets.
In other words, for every "split" of $E$ into two non empty sets, none of them are separated. Even if one split of $E$ is separated, then $E$ is \emph{not connected}.}
\exm{Separated $\implies$ Disjoint, but Disjoint $\not\implies$ Separated.}{$[0,1]$and $(1,2)$ are disjoint, but are not connected since a sequence in $(1,2)$ converges to $1$ in $[0,1]$.}
\thmp{}{$E \subset \RR$ is connected $\iff$ $\forall x,y \in E$, $x<z<y \implies z\in E$.}{$\implies$)
Suppose $\exists x_0,y_0 \in E$ so that $\exists z, x_0<z<y_0$, but $z \not\in E$. Consider $A:=(-\infty,z)$ and $B:=(z,\infty)$.
$A$ and $B$ are seen to be separated, and $E$ is a subset of $A \cup B$, which makes it disconnected.
\\\\ $\impliedby)$ Suppose that we have $E$ disconnected, which means it is the union of two 
separated sets $A$ and $B$ that are non-empty. $x_0 \in A$ and $y_0 \in B$. Consider $z(t)=x_0+t(y_0-x_0)$ for $t\in [0,1]$.
Note that $z(0)=x_0$ and $z(1)=y_0$.
\\\\ Conjecture: There exists a $t_B \in (0,1)$ so that for every $t<t_B$, $z(t)$ does not belong in $B$. If it is not true,
then for every $t \in (0,1)$, there exists a point $t_B<t$ so that $z(t_B)$ is in B. Choose $t=1$ to get $z(t_1)$ in $B$.
Chooe $t=\frac{t_1}{2}$ to get $z(t_2)$ in $B$ with $t_2<t_1$ and $t_2<\frac{1}{2}$. Keep going with $t=\frac{t_{n-1}}{2^{n-1}}$ to get $z(t_n)$ in $B$ with $t_n<t_{n-1}$ and $t_n<\frac{1}{2^{n-1}}$. 
This gives us a sequence $z(t_k)$ which we can see is monotone decreasing assuming $x_0<y_0$. This sequence converges to $z(0)$ which is in $A$ which means that there exists a sequence in $B$, $z(t_n)$ that converges to $A$. Absurd.
\\\\ In a similar vein, we can show that there exists $t_A \in (0,1)$ so that for every $t>t_A$, $z(t)$ is not in $A$. 
Consider $$S_A:=\{ t \in [0,1]: z(t)\in A\cap [x_0,y_0]\}$$ and $$S_B:=\{t\in[0,1]: z(t)\in B\cap [x_0,y_0]\}$$
\\\\ It is easy to see that $S_A$ and $S_B$ are disjoint. If a sequence in one converges in another, say
$t_n \in S_B$ converges to $t_0 \in S_A$. Then $z(t_n)\in B$ by definition, for every $n$.
But then by definition, $z(t_n)=x_0+t_n(y_0-x_0) \in B$ such that
$lim(z(t_n))=x_0+t_0(y_0-x_0) \in A$, which means a sequence in $B$ converges in A. Absurd. So $S_A$ and $S_B$ are separated.
\\\\Note that, for $t>t_A$, no $z(t)$ is in $A$. Hence, we see that for every $t$ so that $z(t)$ falls in $A$, there is an upperbound. Likewise, for every $t$ such that $z(t)$ falls in $B$, there is a lowerbound.
Hence, $S_A$ has a supremum $sup(S_A)$ and $S_B$ has an infimum $inf(S_B)$.
\\\\ At this point, we may as well assume that for every $t<inf(S_B)$, $t \in S_A$ for if not, what we wanted to prove would get proved. 
Suppose then, for argument sake, that for every $t>Sup(S_A)$, $t\in S_B$, and
likewise, for every $t<Inf(S_B)$. $t \in S_A$. Now then, does $Sup(S_A)$ belong
in $S_A$? we see that for every $t>sup(S_A)$, $t \in S_B$ which means we can construct a sequence in $S_B$ using those $t$s, which converge to $Sup(S_A)$ in $S_A$. So that is ruled out.
So is $Sup(S_A)$ in $S_B$? That is not possible either, since $S_A$ is a bounded, infinite set (mainly because supremum isn't in the set), we know that there is a monotone subsequence in $S_A$ converging to $Sup(S_A)$ which is in $S_B$. We therefore conclude that, there exists a point $t \in (0,1)$ so that $t$ is neither in $S_A$, nor in $S_B$. This translates to there being a point $z=z(t)$, between $x_0$ and $y_0$ so that $z(t)\not\in A \cup B \implies \not\in E$. We are, therefore, done. 
\\\\ \textbf{Slicker Argument:} Suppose $E=A \cup B$ with $\bar{A}\cap B=\bar{B} \cap A=\phi$. Consider $x_0 \in A$ and $y_0 \in B$ and WLOG assume $x_0<y_0$. Define $z=sup(A\cap[x_0,y_0])$.
There would be a sequence in $A$ that converges to $z$, by virtue of being the supremum. $z \in \bar{A} \implies z \not\in B$. This means $x_0 \leq z<y_0$. If $z \not\in A$, we would be done.
If $z \in A$, then $z \not\in \bar{B}$. Therefore, $z$ is in an open set $\bar{B}^C$. There would exist
an $\varepsilon_z$-ball around $z$ so that it is fully contained outside $\bar{B}$. Choose $z+\frac{\varepsilon_z}{2}$ as your $z'$.
Note that $z'$ is greater than the supremum of $A$. We see that $z'$ is not in $B$, and not in $A$ either. Hence, we are done.
}

\begin{figure}[h]
    \centering
    \includegraphics[width=0.5\linewidth]{Images/connectednessprof.png}
    \caption{Figure: Proof for the equivalence for connectedness for sets in $\RR$. A look at $A\cap[x_0,y_0]$ and $B\cap [x_0,y_0]$.}
    \label{fig:contced}
\end{figure}

\section{Misc Knowledge}

\thmp{}{\begin{enumerate}
\item If $A$ and $B$ are closed, disjoint subsets of $X$, then $A$ and $B$ are separated.
\item If $A$ and $B$ are open, disjoint sets, then $A$ and $B$ are separated.
\end{enumerate} }{(1) $A$ $B$ closed implies $A=\overline{A}$ and $B=\overline{B}$ which are disjoint. From here it is obvious.
\\\\ (2) $A$ and $B$ are open disjoint sets, then we see that $A \subseteq B^C $ and $B \subseteq A^C$ where 
$A^C$ and $B^C$ are closed by definition. Since closure is the smallest closed set containing $A$ (and $B$), we see that $\overline{A} \subseteq B^C$ and $\overline{B}\subseteq A^C$.
It is now trivial to see that $\overline{A}\cap B \subseteq B^C \cap B=\phi=\overline{B}\cap A $ which is the definition of separated.
}
\corp{Let $p\in X$ and $\delta>0$. Define $A:=B_{\delta}(p)$ and $B=(B_{[\delta]}(p))^C$. $A$ and $B$ are, then, separeted. }{Easy to see that they are both open sets that are disjoint.}{}

\thmp{}{Every connected metric space with atleast two points is uncountable.}{Let $a$ and $b$ be in $X$. Let $\xi\leq d(a,b)$. 
Note that, $P=B_{\xi}(a)$ and $Q=(B_{[\xi]}(a))^C$ are non empty, separated sets (from the previous corollary).
If $X$ is not the union of $P$ and $Q$, then there is a point $z_{\xi}$ in $X$ so that it is neither in $P$
not in $Q$. That means that it is exactly $\xi$ distance away from $p$. For every $\xi<d(a,b)$,
there exists a point $z_{\xi}$ so that its distance from $p$ is exactly $\xi$. Therefore, every $z_{\xi}$ 
is unique (from positivity property of metric spaces) which means there are uncountable $z_{\xi}$-s.
}

\thmp{}{If $P$ and $Q$ are connected such that $P \cap Q \neq \phi$, then $P \cup Q$ is also connected.}{Suppose $P \cup Q$ is actually not connected. This means $P \cup Q= A \cup B$ for non empty, separated sets $A$ and $B$. Suppose $P$ is fully contained in $A$. This means that $Q$ has intersection with $A$ and intersection with $B$ which are non empty. Obviously $Q \subseteq A \cup B$ which means $Q= (A \cap Q)\cup(B \cap Q)$ where $(A \cap Q)$ and $(B \cap Q)$ are separated and non empty. Since $Q$ is connected, this is absurd. Suppose then that $P$ is not fully contained in $A$. This means that $P \cap A$ and $P \cap B$ is non empty each. This means $P=(P \cap A) \cup (P \cap B)$. From the same reasoning, this is absurd.}

\lemp{}{Given two balls $B_1$ and $B_2$ in $\RR^n$ that are closed, with $B_1 \cap B_2=\{z\}$ with $z \in \RR^n$, then the interior of $B_1 \cup B_2$, i.e, $\underline{B_1 \cup B_2}$, is $\underline{B_1} \cup \underline{B_2}$, which are their respective open ball counterparts.}{We understand that $\underline{B_1} \cup \underline{B_2} \subseteq \underline{B_1 \cup B_2}$. Note that none of the "rim" points of $B_1$ or $B_2$, are in the interior. This would conclude the result.}

\corp{If $A \subset X$ is connected, it needn't be true that $\underline{A}$ is connected.}{Consider the set $B_1 \cup B_2$ from the previous lemma. We note that, its interior is the disjoint union of two non empty open balls. These two sets are separated, which makes $\underline{B_1 \cup B_2}$ a separated set.}{}

\thmp{}{Let $E$ be the set of all $x \in [0,1] \subseteq \RR$ so that the decimal expansion of $x$ only contains $4$ and $7$. Then:
\begin{enumerate}
\item $E$ is uncountable
\item $E$ is not dense in $[0,1]$
\item $E$ is compact
\item $E$ is prefect
\end{enumerate}
}{(1) $E$ is uncountable via the diagonal argument. If $\{x_1,x_2,\cdots \}$ is the enumeration of $E$, simply take the first decimal place of $x_1$ and flip it (i.e. to $4$ if $7$ or vice versa). Likewise for $x_2$ and so on to get a new decimal expansion that is unlike all $x_1,x_2\cdots x_n \cdots$ which is a contradiction.
\\\\ (2) Obviously, since $0.4 \leq x \leq 0.8$ for any $x\in E$.
\\\\ (3) We already know $E$ is bounded. Consider $E^c$. This is the set of all numbers in $[0,1]$ so that not all points in the decimal expansion is $4$ or $7$. i.e, there would be a point in the expansion that is neither $4$, nor $7$. Suppose we take one such arbitrary $x \in E^c$. Let the first non $4,7$ number occur at the $j-$th place. $0.z_1z_2z_3\cdots x_j \cdots $. Look for another non $4,7$ after the $j-$th place (if it exists). If it doesn't exist, then all the numbers after $x_j$ would be $4$ or $7$ so safely add $10^{-(j+1)}$ as our $\varepsilon$. This $\varepsilon$ range around $x$ would contain only points of $E^c$. Suppose another point exists that is non $4,7$ after $j$, perhaps at $k>j$-th index. Then it would look something like:
$0.z_1z_2 \cdots z_jx_{j+1}x_{j+2} \cdots z_k \cdots$. Here we simply take $\varepsilon=10^{-(k+1)}$ so that all points in the $\varepsilon$-neighbourhood of $x$ is in $E^c$. Therefore, $E^c$ is open, which means $E$ is closed. Closed and bounded implies compact in $\RR$.
\\\\ We saw that $E$ is closed. We need only show that every point of $E$ is a limit point of $E$. This can be done easily for any $\varepsilon$-ball, using the technique that follows:
Choose a $k$ so that $\frac{1}{10^k}<\varepsilon$. Look at the interval $x-\frac{1}{10^k},x+\frac{1}{10^k}$ that is contained in $E$. Just find some $n>k$, and flip the $4$ to a $7$ or vice versa to land in a "different" element from $x$, yet within the neighbourhood in consideration. Hence, every point is a limit point, making $E$ a perfect set.

}

\thmp{Existence of a compact set in $\RR$ with countable limit points.}{Title}{Consider the points $x_1=1, x_0=0, x_2=\frac{1}{2}, \cdots x_{n}=\frac{1}{n} \cdots$. Let $x_{11}<x_{12} \cdots <x_{1n}$ be a sequence that converges to $x_1=1$. Let $x_{21}<x_{22}<x_{23} \cdots <x_{2n} \cdots $ be a sequence convergent to $x_2=\frac{1}{2}$, with the added condition that $x_{2j} < x_{1k}$ for every $j,k \in \NN$. Likewise for every $x_n$, create a sequence $x_{nk}$ that converges to $x_n$. Make sure that $x_{an}<x_{bm}$ if $a>b$, for every $m,n$. We therefore have:
$$x_{11}<x_{12}\cdots \to x_1 $$  
$$x_{21}<x_{22}\cdots \to x_2$$
$$ \vdots$$
We claim that the set $\Phi=$ $x_0,x_1,x_2, \cdots$ along with $x_{11},x_{12}\cdots, x_{21},x_{22}\cdots x_{n1}$ etc. forms a Compact set in $\RR$ that has countable limit points. 
\\\\ Suppose that $q \in \RR \neq 0$ and $q \neq \frac{1}{j}$ for any $j$ be a limit point of $\Phi$. This means a subsequence $z_n$ in $\Phi$ converges to $q$. Does infinite points of $x_{1j}$ exist in the subsequence $z_n$ convergent to $q$? Obviously not, since that would make a subsequence of $z_n$ convergent to one of $1$. So only utmost finite elements of $x_{1j}$ are in $z_n$. Same way one can argue that utmost finite elements of $x_{kj}$ are in $z_n$ for every $k$. If we establish a subsequence of $z_n$ that converges to $0$, then the only limit point of $\{z_n\}$ would be $0$ which would mean $0$ is where $z_n$ would converge. Enough wishful thinking; Does any point of $x_{1j}$ exist in $z_n$? If yes, choose that point. If not, find the next $n_1$ so that a point in $x_{n_1 j}$ is in $z_n$. Find, then $n_2>n_1$ so that some point in $x_{n_2 j}$ is in $z_n$. As such keep going, making $n_1<n_2<\cdots$ and a sequence that is monotone decreasing by construction, that converges to $0$. Hence, this is a subsequence in $\{z_n\}$ that converges to $0$. If $z_n$ converged to $q$, then the only limit point would be $q$. Hence, $q=0$. This means that the only limit points of $\Phi$ other than $\frac{1}{j}$ is $0$. Hence, we have a countable limit point. Hence, $\Phi$ is a closed set and bounded obviously, with countable limit points. }

\thmp{technique weve already seen}{If $A$ and $B$ are separated sets in $\RR^n$ (that are non empty), and $\vec{x_0} \in A$ and $\vec{y_0} \in B$, define $p(t)=\vec{x_0}+t(\vec y_0 -\vec x_0)$ for $t\in [-\infty,\infty]$ and $$S_A:=\{ t \in \RR: p(t) \in A \} $$ $$S_B:=\{t \in \RR: p(t) \in B \} $$ Then:
\begin{enumerate}
    \item $S_A$ and $S_B$ are separated sets
    \item $\exists t_0 \in (0,1)$ so that $t_0 \not\in S_A \cup S_B$
\end{enumerate}
}{(1) If $S_A$ and $S_B$ weren't disjoint, then obviously $A$ and $B$ wont be. If there is a sequencei n $S_A$ converging in $S_B$ or vice-versa, it is easy to see that this would lead to there exiting a sequence in $A$ converging to $B$ (or vice-versa). Therefore, $S_A$ and $S_B$ are separated.
\\\\ (2) Since $S_A$ and $S_B$ are separated and non-empty, there are two points $x_0$ and $y_0$ in $S_A$ and $S_B$ respectively. Define $z(l)=x_0+l(y_0-x_0)$ for $l \in [0,1]$. Now, for some $l_A$, we have that for every $l>l_A$, $z(l) \not\in S_A$, the set $G_A:=\{l: z(l) \in S_A \}$ is therefore bounded above with supremum $u_A$. Likewise for some $l_B$, we have that for every $l<l_B$, $z(l) \not\in S_B$, the set $G_B:=\{l: z(l) \in S_B \}$ is therefore bounded below with infimum $v_B$. Note that $G_A$ and $G_B$ are separated sets. We may as well assume that for every $l<l_B$, $l$ is in $G_A$, or $z(l)$ falls in $S_A$. If not, we would be done. We now ask: does $v_B$ fall in $G_A$ or $G_B$? If it falls in $G_B$, there exists a sequence in $G_A$ that would converge to $v_B$, which is absurd. If it falls in $G_A$, then by virtue of being the infimum of $G_B$, there is a sequence in $G_B$ converging in $G_A$. Hence, there exists a point $l$ between $x_0$ and $y_0$ $l \not\in G_A$ or $G_B$ which means $z(l) \not \in S_A$ or $S_B$. This again means that $p(z(l)) \not \in A$ or $B$. Phew.   
}

\corp{Every convex set in $\RR^n$ is connected.}{If they were not connected, then there would exist sets $A$ and $B$, non empty, disjoint and separated so that our convex set $C$ would be $A \cup B$. Suppose $x_0 \in A$ and $y_0 \in B$. From the previous theorem we see that, if $z(t)=x_0 +t(y_0-x_0)$, then there would exist $t'$ so that $z(t') \not\in A $ or $B$, which means it wont be in $C$. But if $x_0$ and $y_0$ are in $C$, by definition of convexity, $z(t)$ for any $t$ must exist in $C$. Contradiction. }{}

\exm{A Pedagogical Example.}{If $E:=\{ q \in \QQ: 2<q^2<3\}$ is considered a set in the metric subspace of $\QQ$ with the usual distance, then we have:
\begin{enumerate}
    \item $E$ is closed and bounded (wrt $\QQ$ obviously)
    \item $E$ is \emph{non compact}.
    \item $E$ is open with respect to $\QQ$
\end{enumerate} 
\begin{proof}
Consider an arbitrary convergent sequence $p_n$ in $E$. We see that if $p_n$ is contained in $E$, then $2<p_n^2<3$. If we pass to the limit, we would have $2 \leq q^2 \leq 3$ but the limit $q$ would either not exist in $\QQ$ (whence the sequence $p_n$ wouldn't be convergent anyway) or it does, in which case it follows $2<q^2<3$ (since no rational number has its square as $2$ or $3$). Therefore, we can see that every convergent sequence in $E$ converges in $E$. Hence, $E$ is closed. Boundedness is obvious.
\\\\ One way to see that $E$ is non compact is simply by making use of the "conservation" of compactness going from one space to a bigger space or vice-versa. Since $E$ is clearly not compact in $\RR$, it wont be compact in the metric subspace $\QQ$. Another way to see that it is non compact is to consider the following construction:
Look at the union of the two split intervals $(-\sqrt{3},-\sqrt{2})$ and $(\sqrt{2},\sqrt{3})$. Let $h=\sqrt{3}-\sqrt{2}$. We construct it for the right side interval, and simply copy it to the left one. Choose the point $\sqrt{3}$ and a $h/2$-ball around $\sqrt{3}$ and call it $V_0$. This would intersect the right interval at a distance $h/2$ from $\sqrt{3}$. Choose this intersection point and a $h/4$-ball around this called $V_2$. Both $V_1$ and $V_2$ together covers $3h/4$ of the interval. Choose this intersection point and let the $h/8$-ball around this point be $V_3$. $V_1$, $V_2$ and $V_3$ together cover $7h/8$ of the interval. Keep going as such to construct $V_n$ so that $(V_i)_{i=1}^n$ covers $\frac{(n-1)h}{n}$ of the whole interval. As $n$ tends to $\infty$, the "coverage" converges to $h$. This means that for every $\varepsilon$ distance away, there exists a finite $n_0$ so that $V_1 \to V_{n_0}$ covers up to that $\varepsilon$ distance. Every point $e$ in the strict interval would therefore be covered up by some $V_k$ by our construction. But obviously, this "open cover" has no finite subcover, for no finite "coverage" covers all the way till $h$ distance of the interval. Some $\varepsilon$-gap is always left, hence missing points.
\\\\ $E$ can also be written as the disjoint union of the set $A$ of all $p$ so that $\sqrt{2}<p<\sqrt{3}$ and the set $B$ of all $p$ so that $-\sqrt{3}<p<-\sqrt{2}$. WLOG say $q \in A$. Obviously $\sqrt{2}<q<\sqrt{3}$. Choose $\delta< min\{\sqrt{3}-q, q-\sqrt{2}\}$ and the $\delta-$ball called $V$ around $q$. Easy to see that this ball $V$ is fully contained in $A$. Likewise, it can be shown for $B$ as well. Hence, $E$ is open with respect to $Q$. 
\end{proof}

}
\thmp{Existence of a non-empty, perfect set in $\RR$ that contains no rational points}{Title}{  }

\defn{Separability}{A metric space $X$ is said to be separable if it has a countable dense subset $S$.}
\exm{$\RR^n$ is separable}{Consider the set of all points $z \in \RR^n$ with coordinates in $\QQ$. Basically $\QQ^n$. This would be a countable set, and every point $x \in \RR^n$ is a limit point of this set (since coordinate we can have a rational sequence converging to that coordinate)}

\defn{Basis}{A collection of open sets $\{V_{\alpha}: \alpha \in A\}$ is said to be a basis for $X$, a metric space, if for every point $x \in X$, and for every open set $G$ so that $x \in G$, we have a $V_{\alpha}$ in the basis so that $x \in V_{\alpha} \subset G$. In other words, every open set $G$ can be "covered" by a subcollection of $\{V_{\alpha} \}$}

\thmp{}{For a metric space $X$, separability $\iff$ countable basis}{$\implies$)Suppose $X$ is separable, i.e, has a countably dense set $\{x_1,x_2,\cdots,x_n,\cdots\}$. Choose all the $1-$balls of each of these points to form a countable set. Choose all the $1/2-$ball of each for another countable set. As such keep choosing $1/n$-balls for each $x_j$ in the countably dense set. We claim that this collection of open sets are my basis. Say $z$ is an arbitrary point in $X$, and $G$ an arbitrary open set containing $z$. There would exist $\varepsilon_{z}^G$ so that $\varepsilon_{z}^G$-ball of $z$ would be contained in $G$. Choose $\delta<\frac{\varepsilon_{z}^G}{4}$. There would be an $n$ so that $\frac{1}{n}< \delta$. In this $1/n$ ball around $z$, there would exist a point $x_k$ in the dense set. Choose now, the $1/n$ ball around $x_k$ which contains $z$, and would be completely contained in the $\varepsilon_{z}^G$ ball, which would be contained in $G$. Hence, we are done.
\\\\ $\impliedby$) Say $X$ has a countable basis $\{V_1,V_2,\cdots\}$ that are non empty. Choose $x_1 \in V_1$, $x_2 \in V_2 \cdots$. We claim that this set $\{x_1,x_2 \cdots \}$ forms our countable dense set. Say a point $z \in X$, choose any arbitrary $\varepsilon>0$. The $\varepsilon-$ball around $z$ would be an open set, which means a set $V_k$ is inside $\varepsilon-$ball around $z$.
This would mean that $x_k$ would be in the set, which makes $z$ a limit point of this supposed to be dense set $\{x_1,x_2 \cdots \}$. We are done.

}
\thmp{}{$$"(\exists \{V_{\alpha}: \alpha \in A\})(\forall x \in X)(\forall G \subset X: G\text{ open}: x \in G)(\exists \alpha \in A)(x \in V_{\alpha} \subset G)"$$ $$ \iff $$ $$\exists \{V_{\alpha}: \alpha \in A\} (\forall G \subset X: G \text{ open in }X)(\exists A' \subseteq A)(G=\cup_{\alpha}V_{\alpha}: \alpha \in A') $$ Or in pithy words,
\begin{center}$V_{\alpha}$ is basis if and only if every open set is the union of a subcollection of $\{V_{\alpha} \}$\end{center} }{$\implies$) Consider an open set $G$ (arbitrary) and an arbitrary point $x \in G$. There exists, for every point $x$ in $G$ a corresponding base $V_{\alpha_x}$ so that $x\in V_{\alpha_x} \subset G$. This means, if we collect all the $x$ in $G$ and its corresponding bases, the union of that basis would give us $G$.
\\\\ $\impliedby$) Suppose for any given $G \subset X$ that is open, there is a subcollection of a (fixed) $\{V_{\alpha} \}$ so that $G= \cup_{\alpha'}V_{\alpha'}$. Consider an arbitrary point $x \in X$ and an arbitrary $G: x \in G$. This means there is an epsilon ball around $x$ that is fully contained in $G$. But since open balls are open, there is a subcollection $\{V_{\alpha''}\} $ so that "this ball"$= \cup V_{\alpha''}$. Since $x \in $ this ball, $x \in V_{k}$ for some $k$. But this $v_k$ is in $G$. Hence, we are done.  }
\fact{$$"(\exists \{V_{\alpha}: \alpha \in A\})(\forall x \in X)(\forall G \subset X: G\text{ open}: x \in G)(\exists \alpha \in A)(x \in V_{\alpha} \subset G)"$$ is the same thing as
$$"(\exists \{V_{\alpha}: \alpha \in A\})(\forall G \subset X: G\text{ open})(\forall x \in X: x \in G)(\exists \alpha \in A)(x \in V_{\alpha} \subset G)"$$ }
\defn{}{2nd countable}{A space $X$ is said to be second countable if it has a countable basis.}
\thmp{}{Let $X$ be a metric space in which every infinite subset $E$ has a limit point in $X$. Then $X$ has countable basis (or equivalently is separable).}{Note that no infinite subset of $X$ can be "unbounded" since we have seen before that Limit point compact implies closed and bounded. Consider an arbitrary $\delta_0$. Choose an initial point $x_1$. Is there a point $x_2$ that is outside the $\delta-$ball around $x_1$? If no, the only $\delta-$ball of $x_1$ covers the whole space. If yes, then again ask the question, is there a point $x_3$ that is both outside the $\delta-$ball of $x_1$ and the $\delta-$ball of $x_2$? If not, then these two balls will cover the whole space, if yes, again ask the question, is there a point $x_3$ so that it is not in $x_1$, $x_2$ and $x_3$'s $\delta-$balls. Keep going, as such. We claim that it must terminate after a finite number of steps. Suppose not, i.e for every $x_n$ you can find, you can find an $x_{n+1}$ to form an infinite set of $x_{j}$ such that $x_j$ is not in the $\delta$-ball of any of its preceeding elements $x_{j-1}, x_{j-2} \cdots$. This is true for every $x_j$. From the hypothesis, this set has a limit point $q$. Choose $\varepsilon<\delta/4$, within which exists a point $x_k$ in this infinite set. But since no point exists within the $\delta-$ball of $x_k$ apart from itself, no other point exists in the $\varepsilon-$ball around $x_k$ as well. This would contradict the proposition that $q$ is a limit point. Hence, this procedure must terminate in a finite amount of steps. So for a given $\delta=d$, there exists $n_{\delta}$ balls $V_1,V_2 \cdots V_{n_d}$ with centres $x_1^{d},x_2^d \cdots x_{n_d}^d$. Choose $\delta=1$, and create a collection of 1-balls (by the procedure mentioned above) around the points $x_1^1, x_2^1 \cdots x_{n_1}^1$. Likewise, for $\delta=1/2$, find points $x_2^1, x_2^2 \cdots x_{n_2}^2$, and for $\delta=1/k$, find points $x_2^k, x_2^k \cdots x_{n_k}^k$. Let $z$ be a point in $X$. Choose a $\varepsilon$ ball around $z$. Let $1/n<\varepsilon$. There must be some $x_j^n$ so that $z$ is in the $1/n$-ball of $x_j^n$. Hence, $\{x_j^k: j \in \NN, k \in \NN \}$ is the dense set we need.}
\thmp{}{Separable $\iff$ countable basis $\iff $ every open cover of $X$ has a countable subcover }{$1 \iff 2$) already seen
\\\\ $2 \implies 3)$ Suppose we have an open cover $\{G_{\alpha}: \alpha \in A\}$ of $X$ which has a countable basis $\{V_n\}$. Let $x \in X$ be arbitrary, which means it is in some $G_{\alpha}$, which is the union of some subfamily of $\{V_n\}$ which means $x$ is in some $V_j$ which is inside $G_{\alpha}$. Let $P:=\{n \in \NN: \exists \alpha_{n} \text{such that } V_n \subseteq G_{\alpha_n}\}$. Since every $x$ is in some $V_j$ we have that $X \subseteq \cup_{i=1}^{\infty} V_{n_i} \subseteq \cup_{i \in P} G_{\alpha_i} $ which ultimately means that $G_{\alpha_i}$ forms a countable subcover of $\{G_{\alpha}:\alpha \in A\}$.
\\\\ $3 \implies 1)$ Suppose every open cover has a countable subcover. Consider $\varepsilon=1$ and all $1-$balls in $X$. This is an open cover that has a countable subcover of $1$-balls centred at $\{x_{11},x_{12}, \cdots\}$. Likewise, consider $1/2-$balls in $X$ which would have a countable subcollection of $1/2-$balls centred at $\{x_{21},x_{22},x_{23} \cdots\}$. As such keep going for $\varepsilon=1/n$ to get a countable collection of $1/n$balls. Take all these balls (which gives us a countable collection, from axiom of choice) and call this set $S$. Let $z \in X$. let $\varepsilon>0$ and $1/n<\varepsilon/4$. Consider the $\varepsilon-$ ball around $z$. For $\delta=1/n$ we have a countable collection of points $\{x_{n1},x_{n2} \cdots \}$ whose $1/n$ balls cover the whole space, which means there is a point in $S$, say $x$, whose $1/n$ ball contains $z$. But this ball is in turn in the $\varepsilon-$ball around $z$, which makes $x$ a point in $\varepsilon$ ball of $z$, hence every point of $X$ is a limit point of $S$ (or a point of $S$). }
\thmp{}{A compact metric space has a countable base (or equivalently, is separable- has a countable dense set in it).}{Choose $\delta_1=1$ and all the $\delta_1$-balls in $X$ (centred at every point of $X$). This makes an open cover of $X$ that has a finite subcover. i.e, finite points $x^1_1,x^1_2 \cdots x^1_{n_1} $ whose $\delta_1=1$ balls cover the whole space $X$. Next, choose $\delta_2=1/2$ and every $\delta_2$ ball in $X$. This likewise, would have a finite subcover i.e, $\delta_2$-balls around $x^2_1,x^2_2 \cdots x^2_{n_2}$. As such, for every $\delta_k=1/k$, we have points $x^k_1,x^k_2,\cdots x^k_{n_k}$ whose $\delta_k$ balls cover the whole space.
\\\\ That this space is separable is obvious now. Choose $x^i_{j}: i,j \in \N$. Let $z$ be a point in $X$. Choose any arbitrary $\varepsilon$. Find an $n$ so that $\frac{1}{n}<\varepsilon$. From the result we have, we can always find a $x^k_j$ in our set so that it is in the $\frac{1}{n}$ ball of $z$. This means that in every $\varepsilon$ ball of $z$, there exists a point from our stipulated set. Hence, our set is dense in $X$.  }

\thmp{}{If $X$ is a metric space so that every infinite set in $X$ has a limit point in $X$, then every open cover of $X$ (open relative to $X$) has a finite subcover. Hence:
\begin{center}"If $X$ is a limit point compact metric space, then $X$ is a compact metric space (compact relative to itself, but it also applies to all spaces it lives in, due to preservation of compactness)"
    \\-Krishna, to Arjuna, at Kurukshetra \end{center}
}{Suppose that $X$ is limit point compact. This means that it has a countable basis. Consider an open cover $\{G_{\alpha}: \alpha \in A\}$ so that $X= \cup_{\alpha \in A} G_{\alpha}$. Let $\{V_1,V_2 \cdots\}$ be the countable basis for $X$. This means that for every open set $G$ and every point $x \in X$ so that $x \in G$, we have an $n \in \NN$ so that $x \in V_{n} \subset G$. Let $P:=\{n \in \NN: \exists \alpha \in A \text{ such that } V_n \subseteq G_{\alpha} \}$. Consider any point $x \in X$. For this $x$, there exists some $G_{\alpha}$ so that $x \in G_{\alpha}$ which would imply from the conutable bases property that $\exists V_n$ so that $x\in V_n \subseteq G_{\alpha}$. For every point, there exists a $V_n$ and a corresponding $G_{\alpha}$ so that $x \in V_n \subset G_{\alpha}$. Therefore, $X$ is countably "subcovered" by $G_{\alpha_j}$ (those $G_{\alpha}$ that correspond to the $V_n$, $n \in P$). \\\\ Let this countable subcover be $G_1,G_2 \cdots G_n \cdots $. Suppose to this, there does not exist a finite subcover. i.e, $$\exists x_1 \in G_1^C$$, $$\exists x_2 \in (G_1 \cup G_2)^C=C_1^c \cap G_2^C$$.$$\vdots $$ $$\exists x_n \in (G_1 \cup G_2 \cup G_3 \cup \cdots G_n)^C=\cap_{i=1}^n G_i^C$$, for every $n \in \NN$. Collect all such $x_n$s to form an infinite set in $X$. By hypothesis, this has a limit point, call it $z$. Note that, outside a given $G_k^C$, only finite points of our stipulated set exists, which are possibly $x_1,x_2 \cdots x_{n-1}$. Suppose there exists a $G_{j}^C$ so that $z \not \in G_{j}^C$. This means that $z$ is in $G_j$ which is an open set. Therefore, there exists an $\varepsilon$ ball that is completely inside $G_j$, or rather completely outside $G_j^C$. This means that only finitely many $x_1,x_2 \cdots$ make into this $\varepsilon$-ball of $z$ which is absurd since $z$ is supposedly a limit point. Hence, we cannot have a countable cover that has no finite subcover. Hence, $G_j: j \in \NN$ has a finite subcover, which means $X$ is a compact metric space.}
\rmkb{We showed that every limit point compact \emph{metric space} is compact, i.e, every open cover, open relative to $X$, has a finite subcover. Suppose $X$ is actually a subset of a bigger set $H$ and $X$ inherits the metric from $H$. From conservation of compact relativeness, we see that $X$ must be compact relative to $H$ as well. Hence, every limit point compact set in $X$ is compact in $X$.}
\defn{Haussdrof Space}{A space $X$ in which two distinct points $x$ and $y$ have disjoing neighbourhoods (open sets) $V_x$ and $V_y$ so that $x \in V_x$ and $y \in V_y$ }
\thmp{Equivalent statements pertaining to connectedness}{The following are equivalent:
\begin{enumerate}
\item only clopen sets in $X$ are $X$ and $\emptyset$
\item $X$ is not the disjoint union of two non empty open sets
\item $X$ is not the union of disjoint non empty closed sets
\item $X$ is not the union of separated sets $\overline{A}\cap B=\overline{B}\cap A=\emptyset $
\item Every proper non-empty subset of $X$ has a non empty boundary in $X$
\item $X$ does not have a continuous onto function mapping $X$ to $\{-1,1\}$ with discrete topology
\end{enumerate} }{$1 \implies 2)$ Say $X$ is the disjoint union of two non empty open sets, $X=A \cup B$ where $A$ and $B$ are open. $X \setminus(A)=B$ which, by definition, is closed. But $B$ by hypothesis is open, hence $B$ is a clopen set in $X$ that is not $X$ or $\emptyset$.
\\\\$2 \implies 1)$ Suppose $Z \subset X$ is a non empty clopen, proper subset of $X$. $W=X \setminus(Z)$ is also, by definition, clopen. So we have $W$ and $Z$, disjoint, non empty open sets whose union gives us $X$.
\\\\ $1 \implies 3)$ Suppose $X$ is the disjoint union of two closed sets. $X=A \cup B$. $X \setminus(A)=B$ which is open, but also by hypothesis, closed. Hence, there is a clopen set in $X$.
\\\\ $3 \implies 1)$ Suppose there exists clopen set $Z \subset X$. Then $W=X \setminus(Z)$ is also clopen by definition. Therefore, $W$ and $Z$ form disjoint closed sets whose union gives us $X$.
\\\\ $1 \implies 4)$ Say $X$ is the union of disjoint separated sets $A$ and $B$ so that $\overline{A} \cap B=\overline{B} \cap A=\emptyset$. We then have $B \subseteq (\overline{A})^C$ and likewise $A \subseteq (\overline{B})^C$. $X=A \cup B$. $\overline(A)\subseteq X$ and $\overline{B}\subseteq X$ which means that $\overline{A}\cup B=X$. $\overline{A}\cup B \setminus(\overline{A})=X \setminus(\overline{A})=B$ tells us that $B$ is a open set. Doing the same to $\overline{B} \cup A=X$ tells us that $A$, too, is an open set. Hence, $X$ is the disjoint union of two open sets.
\\\\ $4 \implies 1)$ Say there exists a clopen set $Z \in X$. Consider $Z^C$ which is also clopen, and it is disjoint from $Z$. This means immediately that $\overline{Z} \cap Z^C=Z \cap Z^C =Z \cap \overline{Z^C}=\emptyset$, hence, $X$ is the union of two separated sets.
\\\\ $1 \implies 5)$ Suppose there exists a non empty proper subset of $X$, call it $W$ that has an empty boundary. i.e, $\overline{W}\setminus(W^0)=\emptyset$ which means $\overline{W} \cap (W^0)^C=\emptyset$. $ W^C\subseteq (W^0)^C$ which means $\overline{W} \cup (W^o)^C=X$ which makes $X$ the disjoint union of two closed sets 
\\\\ $5 \implies 1) $ Suppose there exists a clopen set $W$ in $X$. that is neither $X$ nor $\emptyset$. $W^C$ is, then, also clopen. Boundary of $W$ would be $\overline{W} \setminus(W^o)=W \setminus(W)=\emptyset$. Hence, $W$ is a proper set, non empty such that its boundary is empty. 
\\\\ $1 \implies 6)$ Let $f$ be a continuous onto function from $X$ to the discrete set $\{-1,1\}$ with the discrete topology. Consider $\alpha=f^{-1}(-1)$ and $\beta=f^{-1}(1)$. Of course, $\alpha \cup \beta=X$. Is $\alpha$ and $\beta$ disjoint? they must be since $\{-1\}$ and $\{1\}$ are disjoint. Moreover, since $f$ is continuous, and $-1$ and $1$ are open sets in the discrete topology, we have that $\alpha$ and $\beta$ are open sets. Hence, $X$ is the disjoint union of two open sets.
\\\\ $6 \implies 1$) Suppose $X=A \cup B$ where $A $ and $B$ are non empty, disjoint open sets. Define the function $f:X \to \{-1,1\}$ given by $f(x)=-1$ if $x \in A$ and $f(x)=1$ if $x \in B$. $f^{-1}(-1)$ would be $A$ and $f^{-1}(1)$ would be $B$, which means every open set inverse maps to an open set, making $f$ a continuous onto function from $X$ to the discrete topology $\{-1,1\}$ }

\end{document}
